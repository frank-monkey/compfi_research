\documentclass[reqno]{amsart}

\newcommand{\rE}[1]{\tilde{\E}[#1]}


\usepackage{amssymb,amsfonts,amsmath, amsthm}
\usepackage{lipsum} 
\usepackage[all,arc]{xy}
\usepackage{enumerate}
\usepackage{mathrsfs}
\usepackage[usenames, dvipsnames]{xcolor}
\definecolor{cerulean}{RGB}{0,123,167}
\usepackage[colorlinks = true, linkcolor = cerulean, citecolor = Thistle]{hyperref}
\usepackage{tikz}
\usepackage{tikz-cd}
\usepackage[super]{nth}
\usetikzlibrary{arrows}
\usepackage{caption}
\usepackage{geometry}
\geometry{margin = 0.8in, tmargin=0.8in}
\usepackage{multicol}
\usepackage{esint}
\usepackage{mathtools}
\usepackage{import}
\usepackage{xifthen}
\usepackage{pdfpages}
\usepackage{transparent}
\usepackage{upgreek}
\usepackage{pgfplots}
\pgfplotsset{compat=1.17} 
\usepackage{bbm}

\newcommand{\incfig}[1]{
	\fontsize{9pt}{12pt}\selectfont
    \def\svgwidth{\columnwidth}
    \resizebox{0.8\textwidth}{!}{\import{./img/}{#1.pdf_tex}}
}
\newcommand{\C}{\mathbb{C}}
\newcommand{\E}{\mathbb{E}}
\newcommand{\F}{\mathbb{F}}	
\newcommand{\T}{\mathbb{T}}
\newcommand{\R}{\mathbb{R}}
\newcommand{\Q}{\mathbb{Q}}
\newcommand{\N}{\mathbb{N}}
\newcommand{\Z}{\mathbb{Z}}
\newcommand{\K}{\mathbb{K}}
\newcommand{\bH}{\mathbb{H}}
\newcommand{\bP}{\mathbb{P}}
\newcommand{\bS}{\mathbb{S}}
\newcommand{\cO}{\mathcal{O}}
\newcommand{\cU}{\mathcal{U}}
\newcommand{\cV}{\mathcal{V}}
\newcommand{\U}{\mathcal{U}}
\newcommand{\ds}{\displaystyle}
\newcommand{\intbar}{\fint}
\newcommand{\ve}{\varepsilon}
\newcommand{\p}{\partial}
\newcommand{\bvec}{\boldsymbol}
\newcommand{\abs}[1]{\vert#1\vert}
\newcommand{\nab}{\nabla}
\providecommand{\norm}[1]{\left\Vert#1\right\Vert}
\newcommand{\weakar}{\rightharpoonup} 
\newcommand{\weakstarar}{\overset{\ast}{\rightharpoonup}}
\newcommand{\triplenorm}[1]{{\left\vert\kern-0.25ex\left\vert\kern-0.25ex\left\vert #1 
    \right\vert\kern-0.25ex\right\vert\kern-0.25ex\right\vert}}
\newenvironment{sol}
  {\begin{proof}[Solution]}
  {\end{proof}}
\newcommand{\paren}[1]{\left(#1\right)}
\newcommand{\Hzero}{{}_0H^1}
\newcommand{\Hzeros}[1]{{}_0H^{#1}}
\newcommand{\Hzerosig}{{}_0H^1_\sigma}
\renewcommand{\Re}{\operatorname{Re}}
\renewcommand{\Im}{\operatorname{Im}}
\newcommand{\leb}{\mathcal{L}_o^N}
\newcommand{\hauss}{\mathcal{H}_o^1}
\numberwithin{equation}{section}
\newcommand{\lap}[1]{\mathcal{L} \left\{ #1 \right\}}
\newcommand{\lapinv}[1]{\mathcal{L}^{-1}\left\{#1\right\}} 

\DeclareMathOperator{\Span}{span}
\DeclareMathOperator{\esssup}{esssup}
\DeclareMathOperator{\supp}{supp}
\DeclareMathOperator{\curl}{curl}
\DeclareMathOperator{\sgn}{sgn}
\DeclareMathOperator{\lip}{Lip}
\DeclareMathOperator{\epi}{epi}
\DeclareMathOperator{\gr}{Gr}
\DeclareMathOperator{\diam}{diam}
\DeclareMathOperator{\dist}{dist}
\DeclareMathOperator{\meas}{meas}
\DeclareMathOperator{\tr}{tr}
\DeclareMathOperator{\Tr}{Tr}
\DeclareMathOperator{\diverge}{div}
\DeclareMathOperator{\rank}{rank}
\DeclareMathOperator{\proj}{proj}
\DeclareMathOperator{\Range}{Ran}
\DeclareMathOperator{\sym}{sym}


\newtheorem{thm}{Theorem}[section]
\newtheorem{cor}[thm]{Corollary}
\newtheorem{prop}[thm]{Proposition}
\newtheorem{lem}[thm]{Lemma}
\newtheorem{conj}[thm]{Conjecture}
\newtheorem{quest}[thm]{Question}
\theoremstyle{definition}
\newtheorem*{thm*}{Theorem}
\newtheorem*{def*}{Definition}
\newtheorem*{prop*}{Proposition}
\newtheorem{defn}[thm]{Definition}
\newtheorem{defns}[thm]{Definitions}
\newtheorem{nota}[thm]{Notation}
\newtheorem{con}[thm]{Construction}
\newtheorem{exmp}[thm]{Example}
\newtheorem{exmps}[thm]{Examples}
\newtheorem{notn}[thm]{Notation}
\newtheorem{notns}[thm]{Notations}
\newtheorem{addm}[thm]{Addendum}
\newtheorem{exer}[thm]{Exercise}
\newtheorem{prob}[thm]{Problem}
\theoremstyle{remark}
\newtheorem{rem}[thm]{Remark}
\newtheorem{rems}[thm]{Remarks}
\newtheorem{warn}[thm]{Warning}
\newtheorem{sch}[thm]{Scholium}


\newenvironment{nouppercase}{
  \let\uppercase\relax
  \renewcommand{\uppercasenonmath}[1]{}}{}
\newcommand{\rvline}{\hspace*{-\arraycolsep}\vline\hspace*{-\arraycolsep}}
\renewcommand{\P}{\mathbb{P}}
\newcommand{\D}{\mathcal{D}}
\newcommand{\imp}{\;\Rightarrow\;}
\newcommand{\m}{\mathrm}
\newcommand{\opl}{\oplus}
\newcommand{\ot}{\otimes}
\newcommand{\lv}{\lVert}
\newcommand{\rv}{\rVert}
\newcommand{\Lm}{\Lambda}
\newcommand{\J}{\boldsymbol{J}}
\newcommand{\lm}{\lambda}
\newcommand{\al}{\alpha}
\newcommand{\be}{\beta}
\newcommand{\es}{\varnothing}
\newcommand{\lra}{\;\Leftrightarrow\;}
\newcommand{\ep}{\varepsilon}
\newcommand{\f}{\frac}
\newcommand{\sig}{\sigma}
\newcommand{\gam}{\gamma}
\newcommand{\del}{\delta}
\newcommand{\thet}{\vartheta}
\newcommand{\bn}{\binom}
\newcommand{\oo}{^\circ}
\newcommand{\pd}{\partial}
\newcommand{\lf}{\lfloor}
\newcommand{\rf}{\rfloor}
\newcommand{\grad}{\nabla}
\newcommand{\bpm}{\begin{pmatrix}}
\newcommand{\epm}{\end{pmatrix}}
\newcommand{\para}{\parallel}
\newcommand{\loc}{\m{loc}}
\renewcommand{\bar}{\overline}
\newcommand{\Le}{\mathfrak{L}}
\newcommand{\Leo}{\mathfrak{L}_{\m{o}}}
\newcommand{\emb}{\hookrightarrow}
\newcommand{\res}{\restriction}
\renewcommand{\le}{\leqslant}
\renewcommand{\ge}{\geqslant}
\newcommand{\jump}[1]{\left\llbracket#1\right\rrbracket}
\newcommand{\tjump}[1]{\llbracket#1\rrbracket}
\newcommand{\sjump}[1]{\big\llbracket#1\big\rrbracket}
\newcommand{\bjump}[1]{\Big\llbracket#1\Big\rrbracket}
\newcommand{\bnorm}[1]{\Big\lv#1\Big\rv}
\newcommand{\snorm}[1]{\big\lv#1\big\rv}
\newcommand{\tnorm}[1]{\lv#1\rv}
\newcommand{\bp}[1]{\Big(#1\Big)}
\renewcommand{\sp}[1]{\big(#1\big)}
\newcommand{\tp}[1]{(#1)}
\newcommand{\babs}[1]{\Big|#1\Big|}
\newcommand{\sabs}[1]{\big|#1\big|}
\newcommand{\tabs}[1]{|#1|}
\renewcommand{\sb}[1]{\left[{#1}\right]}
\newcommand{\bsb}[1]{\Big[{#1}\Big]}
\newcommand{\ssb}[1]{\big[{#1}\big]}
\newcommand{\tsb}[1]{[{#1}]}
\newcommand{\cb}[1]{\left\{{#1}\right\}}
\newcommand{\scb}[1]{\big\{{#1}\big\}}
\newcommand{\bcb}[1]{\Big\{{#1}\Big\}}
\newcommand{\tcb}[1]{\{{#1}\}}
\newcommand{\br}[1]{\left\langle #1 \right\rangle}
\providecommand{\bbr}[1]{\Big\langle #1 \Big\rangle}
\providecommand{\sbr}[1]{\big\langle #1 \big\rangle}
\providecommand{\tbr}[1]{\langle #1 \rangle}
\newcommand{\ssum}[2]{{\textstyle\sum\limits_{#1}^{#2}}}
\newcommand{\z}[1]{\mathring{#1}}
\newcommand{\ip}[2]{\p{{#1},{#2}}}
\renewcommand{\bf}[1]{\mathbf{#1}}
\newcommand{\ii}{\m{i}}
\DeclareMathOperator{\Div}{div}
\DeclareMathOperator{\ten}{ten}
\DeclareMathOperator{\card}{card}
\pdfstringdefDisableCommands{\def\eqref#1{(\ref{#1})}}


\title{Pricing Arithmetic Asian Options under the Bachelier Model} %Discuss title change? FIXME

\author{Jessica Chen}
\address{
Carnegie Mellon University\\
Pittsburgh, PA 15213, USA
}
\email[J. Chen]{jschen2@andrew.cmu.edu}

\author{Linxuan Jiang}
\address{
Carnegie Mellon University\\
Pittsburgh, PA 15213, USA
}
\email[L. Jiang]{linxuanj@andrew.cmu.edu}

\author{Frank Sacco}
\address{
Carnegie Mellon University\\
Pittsburgh, PA 15213, USA
}
\email[F. Sacco]{fsacco@andrew.cmu.edu}

\author{Albert Zhang}
\address{
Carnegie Mellon University\\
Pittsburgh, PA 15213, USA
}
\email[A. Zhang]{albertzh@andrew.cmu.edu}

\thanks{J. Chen, L. Jiang, F. Sacco, A. Zhang were supported by the MFSURP program at Carnegie Mellon University.}

\keywords{Asian Option, Chooser Option, Bachelier Model, Exotic Option, Option Pricing, Quantitative Finance.}

\begin{document}

\begin{abstract}
     In this set of notes we derive the time-zero prices of various \emph{chooser options} under the continuous Bachelier model.
     These are contracts with a fixed maturity date $T$ and a chooser date $\tau$ satisfying $0 \le \tau \le T$, for which an agent is allowed to choose at time $\tau$ the underlying security that determines the structure of the payoff at time $T$.  
     Included is also 2 methods of approximating Asian Options.

     This paper is still in progress, the following are yet to be added:
     \begin{itemize}
          \item Go through all the fixme's and edit appropriately
          \item Add github link to appendix - explain the code
          \item Greeks + calculations
     \end{itemize}
\end{abstract}


\maketitle  
\tableofcontents

\pagebreak
\section{Introduction}
In April 2020, the price of oil futures went negative.
The often used Black-Scholes model, however, is unable to model assets with negative prices, due to its assumption that asset price follows a log-normal distribution.
This reignited interest for the scarcely-used Bachelier model, a similar mathematical model where asset prices follow a normal distribution, with the advantage of being able to handle negative prices (which was considered a limitation at its inception).

\section{The Bachelier Model}
In this paper we work within the context of the \emph{Bachelier model}, where the stock prices $\{S_t\}_{t \ge 0}$ evolves according to 
\begin{align}\label{eq: r not 0}
	 S_t = e^{rt} \left( S_0 + \kappa^{-rt}W_t + \kappa r \int_0^t e^{-rs} W_s \; ds \right),
\end{align}
where $S_0 > 0$ denotes the initial stock price at time 0, $\{W_t\}_{t \ge 0}$ is a Brownian motion under the risk neutral measure $\tilde{\mathbb{P}}$, r is the interest rate, and $\kappa$ is a measure of volatility.
We note to the reader that in the special case when $r = 0$, \eqref{eq: r not 0} reduces to 
\begin{align}\label{eq: r=0}
     S_t = S_0 + \kappa W_t. 
\end{align}

\subsection{European Call when r = 0}
We first consider a European call where the payoff at time $T$ is given by 
\begin{align}
	 C^E_T = (S_T - K)^+
\end{align}
for a fixed strike price $K$. We note that under $\tilde{\mathbb{P}}$, $ W_T \sim N(0, T)$, therefore
\begin{align}\label{eq: S_T dist}
S_T \sim N(S_0, \kappa^2T) \; \text{under the risk neutral measure} \; \tilde{\mathbb{P}}.
\end{align}
According to the risk neutral pricing formula, the time-0 price of this security is given by 
\begin{align}
	 C^E_0 = \tilde{\E}[(S_T - K)^+].
\end{align}
Recall that if we have a random variable $X$ with probability density function $f_X$ under a probability measure $\mathbb{P}$, then the ``law of the unconscious statistician'' tells us that 
\begin{align}
	 \E[g(X)] = \int_{-\infty}^\infty g(x) f_X(x) \; dx.
\end{align}
In our setting, we have 
\begin{align}
	 g(S_T) = (S_T - K)^+,
\end{align}
and the distribution of $S_T$ under $\tilde{\bP}$ as a random variable is given in \eqref{eq: S_T dist}. Therefore, the time-zero price $V_0$ is given by 
\begin{align}\label{eq: call int}
	 C^E_0 = \tilde{\E}[(S_T - K)^+] =  \tilde{\E}[g(S_T)] = \int_{-\infty}^\infty (x-K)^+ \psi(x) \; dx,
\end{align}
where 
\begin{align}
	 \psi(x) = \frac{1}{\nu}\varphi\left(\frac{x-\mu}{\nu}\right), \; \varphi(y) = \frac{1}{\sqrt{2\pi}}e^{-y^2/2}, 
\end{align}
and
\begin{align}
	 \mu = S_0, \; \nu = \kappa \sqrt{T}.
\end{align}
To compute \eqref{eq: call int}, we first note that since $(x-K)^+ = 0$ for $x \le K$, the domain of integration is the set $\{x \mid x \ge K\}$. Now we use the change of variables 
\begin{align}
	 y = -\frac{x-\mu}{\nu} \Longleftrightarrow x = \mu - \nu y,
\end{align}
and we note that since $\nu > 0$, 
\begin{align}
	 x \ge K \Longleftrightarrow \frac{x-\mu}{\nu} \ge \frac{K - \mu}{\nu} \Longleftrightarrow y \le \frac{\mu - K}{\nu} =: d_-.
\end{align}
Then by performing a change of variables, \eqref{eq: call int} becomes 
\begin{align}
	 C^E_0 = \int_{-\infty}^{d_-} (\nu y + K - \mu) \varphi(-y) \; dy = \int_{-\infty}^{d_-} (\nu y + K - \mu) \varphi(y) \; dy = \underbrace{\int_{-\infty}^{d_-} \nu y \varphi(y) \; dy}_{:= I} + \underbrace{\int_{-\infty}^{d_-}  (K-\mu)\varphi(y) \; dy}_{:= II}.
\end{align}
We define the cumulative distribution function of a standard normal random variable $X$ under $\mathbb{P}$ via 
\begin{align}
	 \varphi(x) = \mathbb{P}[X \le x] = \E[ \mathbbm{1}_{X \le x}] = \int_{-\infty}^x \varphi(y) \; dy.
\end{align}
With this notation in hand, we can write 
\begin{align}
	 II = (K-\mu) \int_{-\infty}^{d_-} \varphi(y) \; dy = (K-\mu) \varphi(d_-),
\end{align}
and 
\begin{align}
	 I = \nu \int_{-\infty}^{d_-} y \varphi(y) \; dy =  \frac{\nu}{\sqrt{2\pi}} \lim_{t \to -\infty} (e^{-t^2/2} - e^{-d_-^2/2}) = -\frac{\nu}{\sqrt{2\pi}} e^{-d_-^2/2}.
\end{align}
Therefore 
\begin{align} \label{European Call r=0}
	 C^E_0 =  -\frac{\nu}{\sqrt{2\pi}} e^{-d_-^2/2} + (K-\mu) \varphi(d_-).
\end{align}
where 
\begin{align}
     \mu = S_0, \; \nu = \kappa \sqrt{T}, \; d_- = \frac{\mu - K}{\nu}.
\end{align}

\subsection{European Put when r = 0}
To compute the price of a put, one can use put-call parity \eqref{Put-Call Parity European}. 
By substitution, 
\begin{align}
     P^E_0 = K - S_0 + (K-\mu) \varphi(d_-) - \frac{\nu}{\sqrt{2\pi}} e^{-d_-^2/2}.
\end{align}
where 
\begin{align}
	 \mu = S_0, \; \nu = \kappa \sqrt{T}, \; d_- = \frac{\mu - K}{\nu}.
\end{align}

\subsection{Arithmetic Asian Call}
Next we consider an \emph{arithmetic Asian call} where the payoff at time $T$ is given by 
\begin{align} \label{eq: asian}
	 C^A_T = (A_T - K)^T, \; A_T = \frac{1}{T}\int_0^T S_t \; dt = S_0 + \frac{\kappa}{T} \int_0^T W_t \; dt. 
\end{align}
As shown in \eqref{BM Appendix}, under the risk neutral measure $\tilde{\mathbb{P}}$, 
\begin{align}
	 \int_0^T W_t \; dt \sim N(0, T^3/3).
\end{align}
Because $\kappa$ is constant we can conclude 
\begin{align}
	 A_T \sim N(S_0, \kappa^2 T/3) \; \text{under the risk neutral measure} \; \tilde{\mathbb{P}}.
\end{align}
Comparing this to \eqref{eq: S_T dist}, we see that $A_T$ has a similar distribution, the only difference is that the variance of $A_T$ is smaller by a factor of $\frac{1}{3}$,
so the standard deviation of $A_T$ is smaller by a factor of $\frac{1}{\sqrt{3}}$. By performing the exact same set of calculation as before, the time-0 price of an Asian option is 
\begin{align} \label{eq: asian op}
	 C^A_0 = -\frac{\nu}{\sqrt{6\pi}} e^{-3d_-^2/2} + (K-\mu) \varphi(\sqrt{3}d_-),
\end{align}
where 
\begin{align}
	 \mu = S_0, \; \nu = \kappa \sqrt{T}, \; d_- = \frac{\mu - K}{\nu}.
\end{align}
We note that since $\sqrt{3} > 1$, we see from \eqref{eq: asian op} that the price of an Asian call is higher than the price of a European call. This should be expected as the taker is paying a premium for a less volatile product. 


\subsection{Arithmetic Asian Put}
To compute the price of a put, one can use the Asian option put-call parity shown \eqref{Put-Call Parity Asian}. By substitution, 

\begin{align}
     P^A_0 = Ke^{-rT} + (K-\mu) \varphi(\sqrt{3}d_-) - \frac{\nu}{\sqrt{6\pi}} e^{-3d_-^2/2} - w_0.
\end{align}
where 
\begin{align}
	 \mu = S_0, \; \nu = \kappa \sqrt{T}, \; d_- = \frac{\mu - K}{\nu}.
\end{align}
Recall that $w_0$ is the price of an option at time 0 which pays $A_T$ at time $T$.

\section{Chooser Pricing under the Bachelier Model}
In this section we derive the arbitrage-free prices of exotic "chooser" contracts.
\subsection{Properties of a Chooser}
In this section we consider a more complicated type of financial contracts known as \emph{chooser options}. 
These are contracts with a fixed maturity date $T$ and a strike price $K$, and an agent is allowed to decide on a choosing date $\tau < T$ to choose the underlying derivative in the contract.
Many results are known when an agent is allowed to choose between a European call and a European put; we are interested in a variant of this type of contract that allows an agent to decide between two securities that pays 
\begin{align}
     C_T = (A_T - K)^+, \; P_T = (K - A_T)^+,
\end{align}
where $A_T$ is defined via \eqref{eq: asian}.
Here, we assume the agent chooses optimally with no outside information. At time $\tau$, the agent will choose the option of higher value between the put and call, therefore the value of this contract at time $\tau$ is 
\begin{align}\label{eq: V_tau}
     V_\tau = \max(C_\tau, P_\tau).
\end{align}
The time-zero price of this contract is then 
\begin{align}
      V_0 = e^{-r\tau}\tilde{\E}[V_\tau].
\end{align}
In the next subsection, we simplify the expression for $V_\tau$ via the method of replication. 
\subsection{Replication}
We first note that by properties of the max function \eqref{Max appendix}, we can write 
\begin{align}\label{eq: chooser formula PC}
     V_\tau = C_\tau + \max(0, P_\tau - C_\tau)
\end{align}
where $P_\tau$ and $C_\tau$ are an Asian put and call, respectively.
By \eqref{eq: pos part decomp}, we have
\begin{align}\label{eq: P - C}
     P_T - C_T = (K-A_T)^+ - (A_T - K)^+ = K - A_T. 
\end{align}

Next, we identify the time-$\tau$ prices of contracts paying $P_T - C_T$ and $K - A_T$ at time $T$.

To replicate a security with payoff $P_T - C_T$, we consider a portfolio that goes long an Asian Put and short an Asian Call at time $0$, both with maturity $T$ and strike $K$.
To replicate a security with payoff $K - A_T$,  we consider a portfolio investing $Ke^{-rT}$ into the money account at time 0 and shorting a contract (which we will identify later), which pays $A_T$ at time $T$. 

Since both portfolios have the same payoff at time $T$ by \eqref{eq: P - C},
they have the same price for all times $t$ where $0 \leq t \leq T$ under the assumption that the market is
arbitrage-free according to \ref{Arbitrage-Free Pricing Appendix}.

%For any $t$ satisfying $0 \le t \le T$, we define $P_t$ to be the time-$t$ value of a Put with payoff $P_T$ at time $T$,
%$C_t$ to be the time-$t$ value of a call with payoff $C_T$ at time $t$, $w_t$ to be the time-$t$ value of an Asian option paying $A_T$ at time $t$.

Using this notation, at time $\tau$ the value of the first portfolio is $P_\tau - C_\tau$.
 Also, at time $\tau$ the second portfolio has $Ke^{-rT + r\tau}$ in the bank and is shorting a contract which pays $A_T$ at $T$, therefore the time-$\tau$ value of the second portfolio is $Ke^{r(\tau - T)} - A_\tau$. We denote the value of a contract at time $\tau$ which pays $A_T$ at time $T$ as $w_\tau$.
By replication, the time $\tau$ prices of the portfolios are equal, therefore we have 
\begin{align}\label{eq: P-C Parity}
     P_\tau - C_\tau = Ke^{r(\tau - T)} - w_\tau.
\end{align}
Substituting this result back into \eqref{eq: V_tau}, the value of the original chooser contract at $\tau$ is
\begin{align} \label{V_tau Asian Chooser}
     V_\tau = C_\tau + \max(0, Ke^{r(\tau - T)} - w_\tau).
\end{align}
Our next goal is to find an explicit formula for $w_\tau$. 

For simplicity, we define $U_\tau$ to be the time $\tau$ price of a contract with payoff $Y_T$ at time $T$, where $Y_T$ is defined via
\begin{align}\label{eq: y}
     Y_T = \int_0^T S_t \; dt.
\end{align}
Once $U_\tau$ is determined, then we can recover $w_\tau$ as $w_\tau = \dfrac{U_\tau}{T}$. 

Note that \eqref{eq: y} can be split into
two parts, 
\begin{align}
     Y_T = \int_0^\tau S_t \; dt + \int_\tau^T S_t \; dt.
\end{align}
Observe that the integral from $0$ to $\tau$ is known at time $\tau$ as each price $S_t$ will be known by the time $\tau$. So we can treat this integral as a constant and now try to replicate the integral from time $\tau$ to $T$.

\subsection{Replicating the Asian chooser when $r > 0$}
We begin our replicating strategy by buying $x$ shares of stock at time $\tau$. For all times $t$ where $\tau \leq t \leq T$, we will continuously sell off stock at the rate $\alpha_t$ and invest the revenue.
With this strategy, at time $T$, the bank has 
\begin{align}
     \int_\tau^T \alpha_t S_t e^{r(T-t)} \; dt
\end{align}
To finish the replication, we want our replicating portfolio to be equal to the integral we are replicating:
\begin{align}
     \int_\tau^T \alpha_t S_t e^{r(T-t)} \; dt = \int_\tau^T S_t \; dt.
\end{align}
Solving for $\alpha_t$, we find that
\begin{align}
     \alpha_t = e^{r(t-T)}
\end{align}
Thus, the amount of shares our strategy started with was
\begin{align}
     x = \int_\tau^T e^{r(t-T)} \; dt = \dfrac{1}{r} - \dfrac{e^{r(\tau - T)}}{r}.
\end{align}
This tells us that the cost at time $\tau$ to receive the stock from times $\tau$ to $T$ continuously is $x S_\tau$. This gives us 
\begin{align}
     U_\tau = \int_0^\tau S_t \; dt + \dfrac{S_\tau}{r}\left( 1 - e^{r(\tau - T)} \right)
\end{align}
Recall that $w_\tau$ is the price at time $\tau$ to receive $A_T$, equivalent to $\dfrac{Y_T}{T}$, at time $T$.
Thus, the price at $\tau$ to receive just $A_T$ is 
\begin{align}
     w_\tau = \dfrac{U_\tau}{T} = \dfrac{\int_0^\tau S_t \; dt + \dfrac{S_\tau}{r}\left( 1 - e^{r(\tau - T)} \right)}{T}
\end{align}
Returning to \eqref{eq: P-C Parity}, we can write out the equation as
\begin{align}
     P_\tau - C_\tau = Ke^{r(\tau - T)} - \dfrac{\int_0^\tau S_t \; dt + \dfrac{S_\tau}{r}\left( 1 - e^{r(\tau - T)} \right)}{T}.
\end{align}
Substituting this into \eqref{V_tau Asian Chooser}, the value of $V_\tau$ is
\begin{align}
     V_\tau = C_\tau + \max(0, Ke^{r(\tau - T)} - \dfrac{\int_0^\tau S_t \; dt + \dfrac{S_\tau}{r}\left( 1 - e^{r(\tau - T)} \right)}{T})
\end{align}


\subsection{Replicating Asian options when r = 0}
We now consider the case when $r=0$. Observe we cannot plug $r=0$ into the formula we derived for $r>0$ since we divide by $r$.
However, we can apply a similar replication argument as before. (FIXME: this is a little confusing since the interest rate was never specified. maybe specify that you are assuming $r > 0$ initially and now you're considering $r = 0$ as a special case. If you choose to do this I recommend breaking this section into different subsections).
To fix this, we return to our replicating strategy for $w_\tau$ accounting for this special case.

Define $U_\tau$ and $Y_T$ the same way as above. Again, split the integral $Y_T$ such that
\begin{align}
     Y_T = \int_0^\tau S_t \; dt + \int_\tau^T S_t \; dt
\end{align}
We now replicate the integral from time $\tau$ to $T$ for the special case. We follow the same replicating strategy as before. Purchase $x$ shares of stock. For all times $t$ where $\tau \leq t \leq T$, we continuously sell off at the rate $\alpha_t$ and invest the revenue. 
By time $T$, the bank will have 
\begin{align}
     \int_\tau^T \alpha_t S_t \; dt
\end{align}
We finish the replication by setting this equal to the value we're replicating
\begin{align}
     \int_\tau^T \alpha_t S_t \; dt = \int_\tau^T S_t \; dt
\end{align}
Solving for $\alpha_t$, we see that when $r = 0$ that $\alpha_t = 1$. Thus, the number of shares the strategy started with was
\begin{align}
     \int_\tau^T dt = T - \tau
\end{align}
Similar to the $r \neq 0$ case, it then follows that
\begin{align}
     U_\tau = \int_0^\tau S_t dt + S_\tau (T - \tau), \; w_\tau = \dfrac{U_\tau}{T} = \dfrac{\int_0^\tau S_t dt + S_\tau (T - \tau)}{T}
\end{align}
Thus, Put-Call Parity in the special case tells us that 
\begin{align}
     P_\tau - C_\tau = K - \dfrac{U_\tau}{T} = K - \dfrac{\int_0^\tau S_t dt + S_\tau (T - \tau)}{T}.
\end{align}
Substituting this result into the chooser option formula, we have
\begin{align}
     V_\tau = C_\tau + \max(0, K - \dfrac{\int_0^\tau S_t \; dt + S_\tau (T - \tau)}{T})
\end{align}
Note that when the interest rate is $0$, the stock prices evolve according to 
\begin{align}
     S_t = S_0 + \kappa W_t
\end{align}
where $S_0 > 0$ and $\{W_t\}_{t \ge 0}$ is a Brownian Motion under the risk neutral measure.
We can now rewrite our chooser option formula as
\begin{align}
     V_\tau = C_\tau + \max(0, K - \dfrac{\int_0^\tau \left( S_0 + \kappa W_t \right) \; dt + (S_0 + \kappa W_\tau) (T - \tau)}{T})
\end{align}
Simplifying, we find that 
\begin{align}
     V_\tau = C_\tau + \left(K- S_0 - \frac{\kappa(T-\tau)}{T} W_\tau - \frac{\kappa}{T} \int_0^\tau W_t \; dt  \right)^+.
\end{align}
Then by the risk-neutral pricing formula and the linearity of expection, the time-zero price $V_0$ is given by
\begin{align} \label{eq: expect form}
     V_0 = \tilde{\E}[C_\tau] + \tilde{\E}\left[\left(K- S_0 - \frac{\kappa(T-\tau)}{T} W_\tau - \frac{\kappa}{T} \int_0^\tau W_t \; dt  \right)^+\right].
\end{align}
Let $X$ be the random variable defined via 
\begin{align}
     X &= \frac{\kappa(T-\tau)}{T} W_\tau + \frac{\kappa}{T} \int_0^\tau W_t \; dt.
\end{align}
We now calculate the mean and variance of random variable $X$.
We define $X$ as
\begin{align}
     X &= Y + Z
\end{align}
where 
\begin{align}
     Y &= \frac{\kappa(T-\tau)}{T} W_\tau\\
     Z &= \frac{\kappa}{T} \int_0^\tau W_t \; dt.
\end{align}
Note from \eqref{BM Appendix} that the mean of a Brownian Motion is $0$, thus the means of both $Y$ and $Z$ are $0$. 
We now calculate the variance of $X$ as the sum of two random variables
\begin{align}
     Var(X) = Var(Y + Z)
\end{align}
It is known that
\begin{align}
     Var(Y + Z) = Var(Y) + Var(Z) + 2Cov(YZ).
\end{align}
Note from \eqref{BM Appendix} the variances of Brownian Motion. It follows that 
\begin{align}
     Var(Y) &= \tau \left(\frac{\kappa(T-\tau)}{T} \right)^2\\
     Var(Z) &= \frac{\tau^3}{3} \left(\frac{\kappa}{T}\right)^2
\end{align}
To calculate the covariance term, we expand it out in terms of expected value.
Note that the expected values of a Brownian Motion is $0$ so $\E(Y) = \E(Z) = 0$, therefore
\begin{align}
     Cov(YZ) = \E(YZ) - \E(Y)\E(Z) = \E(YZ)
\end{align}
Substituting, we can now rewrite the covariance as
\begin{align}
     Cov(YZ) = \E(W_\tau \int_0^\tau W_t \; dt) \;\dfrac{\kappa^2 (T - \tau)}{T^2}
\end{align}
For simplicity, let $\alpha = \dfrac{\kappa^2 (T - \tau)}{T^2}$.
By a property of integrals and expected value, we can move the integral outside the expected value as such (FIXME:  probably need to fix this).
\begin{align}
     \alpha \E(W_\tau \int_0^\tau W_t \; dt) = \alpha \int_0^\tau \E(w_\tau w_t) dt
\end{align}
Observe that $t \leq \tau$. Thus, we can further simplify down to
\begin{align}
     \alpha \int_0^\tau \E((w_\tau + w_t - w_t) w_t) dt = \alpha \int_0^\tau \E(w_t^2 + (w_\tau - w_t) w_t) dt
\end{align}
We can expand the expected value by linearity of expectations. 
Recall from \eqref{BM Appendix} that the expected value of $(w_\tau - w_t) w_t$ is $0$ and that the expected value of $w_t^2$ is $t$. Thus, we have
\begin{align}
     Cov(XY) = \alpha \int_0^\tau t \; dt = \alpha \dfrac{\tau^2}{2}
\end{align}
It follows that the mean and variance of $X$ can be computed as (IMPORTANT FIXME: add details of this computation!)
\begin{align}
     \mu &= \E(X) = 0\\
     \sigma^2 &= Var(X) = \tau(\frac{\kappa(T-\tau)}{T})^2 + \frac{\tau^3}{3}(\frac{\kappa}{T})^2 + \tau^2\frac{\kappa^2(T-\tau)}{T^2}.\\
     \nu &= \sigma
\end{align}
(FIXME: typically $\sigma$ denotes the standard deviation, so this isn't very consistent with standard notation)

Using methods of stochastic calculus detailed in \eqref{BM Appendix}, 
we know $X \sim N(\mu,\sigma^2)$.
Therefore we can define the probability density function as
\begin{align}
     \label{eq: phi-PDF}
     \varphi(x) &=\frac{1}{\sqrt{2\pi}}e^{-\frac{x^2}{2}}\\
     \psi(x) &= \frac{1}{\nu}\varphi(\frac{x-\mu}{\nu})
\end{align}
Now we can substitute back into our equation from \eqref{eq: expect form} and use the definition of expectation on a continuous random variable to get 
\begin{align} \label{eq: prechange}
     V_0 = \tilde{\E}[C_\tau] + \int_{-\infty}^{\infty}\left(K- S_0 - x\right)^+ \psi(x) \; dx.
\end{align}
To integrate the second term in $V_0$ we will let
\begin{align}
     z &= \frac{x-\mu}{\sigma}.
\end{align}
it follows that 
\begin{align}\label{eq: change variable}
     x &= z\nu + \mu\\
     dx &= \nu dz
\end{align}
We note that 
\begin{align} \label{eq: int domain}
    K - S_0 - x \ge 0 \Longleftrightarrow x \leq K- S_0 \Longleftrightarrow  \frac{x-\mu}{\sigma} \leq \frac{K-S_0-\mu}{\nu}
\end{align}
and define $d_-$ via
\begin{align}
      d_- =  \frac{K-S_0-\mu}{\nu}
\end{align}
so by \eqref{eq: int domain}, we have 
\begin{align} \label{eq: change bound}
     K - S_0 - x \ge 0  \Longleftrightarrow z \leq d_.
\end{align}
Now using \eqref{eq: change variable} and \eqref{eq: change bound} we can rearrange \eqref{eq: prechange} as:
\begin{align}
     V_0 = \tilde{\E}[C_\tau] + \left(\int_{-\infty}^{d\_}\left(K- S_0 - z\nu - \mu\right)\varphi(z) \; dz\right).
\end{align}
Simplifying this expression we get the form
\begin{align}
     V_0 = \tilde{\E}[C_\tau] - \nu \int_{-\infty}^{d_-}z\varphi(z)\nu \; dz +  (K- S_0 - \mu)\int_{-\infty}^{d_-}\varphi(z) \; dz 
\end{align}
Excluding the call, we can easily simplify the latter of the terms by defining the cumulative distribution function
\begin{align} \label{eq: cdfnormal}
     \Phi(x) = \int_{-\infty}^x \varphi(y) dy.
\end{align}
We resolve the former term by first substituting in \eqref{eq: phi-PDF}
\begin{align}
     \dfrac{-\nu}{\sqrt{2\pi}} \;\int_{-\infty}^{d_{-}} ye^{-y^2} = \dfrac{\nu}{\sqrt{2\pi}} \;e^{\frac{-d_-^2}{2}}
\end{align}
Thus, we find that
\begin{align}
     V_0 = \tilde{\E}[C_\tau] + \dfrac{\nu}{\sqrt{2\pi}} \;e^{\frac{-d_-^2}{2}} +  (K- S_0 - \mu) \Phi(d_-),
\end{align}
which through \eqref{replicating puts/calls}, gives us the final equation.
\begin{align}
     V_0 = C_0 + \dfrac{\nu}{\sqrt{2\pi}} \;e^{\frac{-d_-^2}{2}} +  (K- S_0 - \mu) \Phi(d_-),
\end{align}
where
\begin{align}
      d_- =  \frac{K-S_0-\mu}{\nu}
\end{align}

\section{Chooser Option Variants}

\subsection{Tail Chooser}
We will now consider a variant of the Asian chooser we looked at earlier. We assume all conditions remain the same, except we now define $A_{\tau, T}$ as
\begin{align}
     A_{\tau, T} = \int_\tau^T S_t \;dt
\end{align}
where $\tau$ is the choice date, and $T$ is the time of maturity.

\subsection{Asian Tail Choosers when $r = 0$}
To price this option, we slightly modify the replication strategy from before. Let $Y_T = \int_\tau^T S_t \;dt$ and $U_\tau$ be the price at $\tau$ to receive $Y_T$ at time $T$.

We proceed with the replication of $Y_T$. Suppose an agent purchases $x$ shares at time $\tau$, and chooses to sell them off continuously at rate $\alpha_t$ at time $t$.
At time $T$, the agent's portfolio is worth
\begin{align}
     \int_\tau^T \alpha_t S_t e^{r(T-t)} \;dt
\end{align}
Since we assume here that $r=0$, this reduces to
\begin{align}
     \int_\tau^T \alpha_t S_t \;dt
\end{align}
To complete the replication, we set this equal to the value we are trying to reproduce:
\begin{align}
     \int_\tau^T \alpha_t S_t \;dt = \int_\tau^T S_t \;dt
\end{align}
It follows that
\begin{align}
     \alpha_t = 1
\end{align}
for all $t$ where $\tau \leq t \leq T$. Thus,
\begin{align}
     x = \int_\tau^T \alpha_t \;dt = T - \tau.
\end{align}
It then follows that $U_\tau = (T - \tau) S_\tau$. Observe that 
\begin{align}
     A_{\tau, T} = \dfrac{U_\tau}{T - \tau}.
\end{align}
Again using the notation $w_\tau$ as the price needed at time $\tau$ to receive $A_{\tau, T}$ at time $T$, it follows that
\begin{align}
     w_\tau = \dfrac{(T - \tau) S_\tau}{T - \tau} = S_\tau.
\end{align}
Referring back to \eqref{eq: P-C Parity} and using $r = 0$, we have
\begin{align}
     P_\tau - C_\tau = K - S_\tau.
\end{align}
Using \eqref{eq: chooser formula PC}, the price of the tail chooser option with choice date $\tau$, which we write as $V_\tau$, is
\begin{align}
     V_\tau = C_\tau + (K - S_\tau)^+.
\end{align}
Recall in the Bachelier model that the stock evolves according to $S_t = S_0 + \kappa W_t$ when $r = 0$, $S_0 > 0$, and $W_t$ is a Brownian Motion under the risk-neutral measure. Then,
\begin{align}
     V_\tau = C_\tau + (K - (S_0 + \kappa W_\tau))^+.
\end{align}
Applying the risk-neutral pricing formula and linearity of expectations, we have
\begin{align} \label{eq: expect form}
     V_0 = \tilde{\E}[C_\tau] + \tilde{\E}((K - S_0 - \kappa W_\tau)^+).
\end{align}
To simplify the above, define random variable $X$ and function $g(X)$ as 
\begin{align} 
     X = \kappa W_\tau, \;g(X) = (k - S_0 - X)^+
\end{align}
Applying the law of the unconscious statistician, we can express $V_0$ as 
\begin{align} \label{eq: gxtailint}
     V_0 = \tilde{\E}[C_\tau] + \int_{-\infty}^{\infty} (k - S_0 - X)^+ \psi(x) dx
\end{align}
where
\begin{align} 
     \psi(x) = \dfrac{1}{\nu} \varphi \left( \dfrac{x - \mu}{\nu} \right)\\
     \label{eq: pdfphieq}
     \varphi(x) = \dfrac{1}{\sqrt{2\pi}} e^\frac{-x^2}{2}
\end{align}
Let $y = \dfrac{x - \mu}{\nu}$. Observe that
\begin{align} \label{eq: ysubst}
     y = \dfrac{x - \mu}{\nu} \implies x = y\nu + \mu \implies dx = \nu \;dy
\end{align}
We can now take the positive part of the integral from \eqref{eq: gxtailint}
\begin{align}
     k - S_0 - x \geq 0 \implies -x \geq S_0 - k
\end{align}
Adding $\mu$ and dividing by $\nu$ on both sides,
\begin{align}
     -y = \dfrac{-x + \mu}{\nu} \geq \dfrac{S_0 - k + \mu}{\nu}
\end{align}
It follows that 
\begin{align}
     y \leq \dfrac{k - S_0 - \mu}{\nu} = d_-
\end{align}
We now evaluate \refeq{eq: gxtailint} using \refeq{eq: ysubst}
\begin{align} 
     V_0 = \tilde{\E}[C_\tau] + \int_{-\infty}^{d_-} (k - S_0 - y\nu - \mu) (\dfrac{1}{\nu} \varphi(y)) (-\nu) \;dy
\end{align}
Simplifying and splitting the integral, we have
\begin{align} 
     V_0 = \tilde{\E}[C_\tau] - \int_{-\infty}^{d_-} (k - S_0 - \mu) \varphi(y) \;dy + \int_{-\infty}^{d_-} y\nu \varphi(y)) \;dy
\end{align}
Define the CDF the same as \eqref{eq: cdfnormal}. 
\begin{align} 
     V_0 = \tilde{\E}[C_\tau] - (k - S_0 - \mu) \Phi(d_-) + \int_{-\infty}^{d_-} y\nu \varphi(y) \;dy
\end{align}
The remaining integral term can be simplified through \eqref{eq: pdfphieq}.
\begin{align} 
     \nu \int_{-\infty}^{d_-} y \varphi(y) \;dy = \nu \int_{-\infty}^{d_-} y \dfrac{1}{\sqrt{2\pi}} e^{\frac{-y^2}{2}} \;dy = -\dfrac{\nu}{\sqrt{2\pi}} e^{\frac{-d_-^2}{2}}
\end{align}
Substituting we get
\begin{align} 
     V_0 = \tilde{\E}[C_\tau] - (k - S_0 - \mu) \Phi(d_-) -\dfrac{\nu}{\sqrt{2\pi}} e^{\frac{-d_-^2}{2}}
\end{align}
We can then substite $C_\tau$ from \eqref{European Call r=0} to get FIXME.

\section{Approximating Asian Options}
It may not be feasible to compute an Asian Option in a short time span.
Following are some approximations which sacrifice accuracy to reduce computations and increase speed.

\subsection{Approximating Arithmetic Asian under Black-Scholes Model}
(FIXME: find reference for why this is an acceptable approximation, flesh out reasoning)
In the Black-Scholes Model, the asset price $S_t$ for $0\leq t \leq T$ evolves according to
\begin{align} 
     S_t = S_0e^{\sigma W_t+(r - \frac{1}{2} \sigma^2)t}.
\end{align}
Recall the notation
\begin{align}
     A_T = \frac{1}{T}\int_0^T S_t \; dt.
\end{align}
Note that the integrand $S_t$ is a lognormal random variable.
However, the integral of $S_t$ will not be lognormal (FIXME: cite citation from above).
Thus, we model $A_t$ with a log-normal $Y_t$ with the same mean and variance, which leaves us with equations
\begin{align} \label{expect approx 1}
     \tilde{\E}[A_T] = \tilde{\E}[Y_T] \; \\
     \label{expect approx 2}
     \tilde{\E}[A_T^2] = \tilde{\E}[Y_T^2] \; \\
     Y_t = Y_0e^{\Gamma W_t+rt-\frac{1}{2}\Gamma^2t} \;
\end{align}
Starting with the RHS of \eqref{expect approx 1}
\begin{align} 
     \rE{A_T} 
     = \frac{1}{T} \int_0^T \tilde{\E}[S_t] \; dt 
     = \frac{1}{T} \int_0^T \tilde{\E}[S_0 e^{\sigma W_t+rt-\frac{1}{2}\sigma^2t}] \; dt
     = \frac{S_0}{T} \int_0^T\rE{e^{\sigma W_t}e^{rt-\frac{1}{2}\sigma^2t}} \; dt
\end{align}
We can factor out the constant part of the expected value and are left to evaluate
\begin{align} \label{simplified expected A_T}
     \rE{A_T} = \frac{S_0}{T} \int_0^T e^{rt-\frac{1}{2}\sigma^2t} \rE{e^{\sigma W_t}}  \; dt
\end{align}
Using Moment Generating Function $e^{yX}$ via \eqref{mgf func}, we know if $X \sim N(\mu, \sigma^2)$ the following is true
\begin{align} \label{moment 1}
     \rE{e^{yX}}
     = e^{\mu y + \frac{1}{2} \sigma^2 y^2}
\end{align}
Recalling that the Brownian motion as defined in \eqref{BM Appendix}, $X \sim N(\mu, \sigma^2)$
\begin{align} 
     \rE{e^{\sigma W_t}}
     = e^{\frac{1}{2}\sigma^2 t},
\end{align}
which we can plug in to \eqref{simplified expected A_T} to get
\begin{align} 
     \rE{A_T}
     = \frac{S_0}{T} \int_0^T e^{rt-\frac{1}{2}\sigma^2t} \rE{e^{\sigma W_t}}  \; dt
     = \frac{S_0}{T} \int_0^T e^{\frac{1}{2}\sigma^2 t} e^{rt-\frac{1}{2}\sigma^2t} \; dt
     = \frac{S_0}{T} \int_0^T e^{rt} \; dt
     = \frac{S_0}{rT} (e^{rT}-1)
\end{align}
We can then simplify the LHS of \eqref{expect approx 1} using the moment generating function \eqref{moment 1} to get
\begin{align} 
     \rE{Y_T} 
     = \rE{Y_0e^{\Gamma W_t+rt-\frac{1}{2}\Gamma^2t}} \;
     = \rE{Y_0} \rE{e^{rt - \frac{1}{2}\Gamma^2t}} \rE{\Gamma W_t}
     = Y_0 e^{rt - \frac{1}{2}\Gamma^2t} e^{\frac{1}{2}\Gamma^2 t}
     = Y_0 e^{rT}
\end{align}
So, according to \eqref{expect approx 1}, we have
\begin{align} \label{calibration 1 r neq 0}
     Y_0 = \frac{S_0}{rT}(1-e^{-rT})
\end{align}
We next simplify the second equation. Moving onto the LHS of \eqref{expect approx 2},
\begin{align}
     \rE{A_T^2}
     = {\frac{1}{T^2} \left(\int_0^T \rE{S_t} \; dt\right)^2}
\end{align}
Observe that the square of an integral can be rewritten as a double integral as follows
\begin{align}
     {\frac{1}{T^2} \left(\int_0^T \rE{S_t} \; dt\right)^2}
     = {\frac{1}{T^2} \int_0^T \int_0^T \rE{S_s} \rE{S_t} \; ds \; dt}
     = {\frac{1}{T^2} \int_0^T \int_0^T \rE{S_s S_t} \; ds \; dt}
\end{align}
WLOG, assume that $0 \leq s \leq t$. Integrating under these conditions yields us half of the desired area. Through a symmetry argument, we can conclude that
\begin{align}
     {\frac{1}{T^2} \int_0^T \int_0^T \rE{S_s S_t} \; ds \; dt}
     = {\frac{2}{T^2} \int_0^T \int_0^t \rE{S_s S_t} \; ds \; dt}
\end{align}
We now focus on the expression inside the expected value. 
We can expand $S_s$ and $S_t$ and separate out the Brownian motions
\begin{align}
     S_{s} S_{t} 
     = S_0^2 e^{(r-\frac{1}{2}\sigma^2)(t+s)+\sigma(W_t+W_s)}
     = S_0^2 e^{(r-\frac{1}{2}\sigma^2)(t+s)} e^{\sigma W_t} e^{\sigma W_s}
\end{align}
Observe that $W_t = W_s + (W_t - W_s)$. Using this, we write
\begin{align}
     S_{s} S_{t} 
     = S_0^2 e^{(r-\frac{1}{2}\sigma^2)(t+s)} e^{\sigma W_t} e^{\sigma W_s}
     = S_0^2 e^{(r-\frac{1}{2}\sigma^2)(t+s)} e^{\sigma W_s} e^{\sigma (W_t - W_s)} e^{\sigma W_s}
     = S_0^2 e^{(r-\frac{1}{2}\sigma^2)(t+s)} e^{2 \sigma W_s} e^{\sigma (W_t - W_s)}
\end{align}
This is of interest because the Brownian motions $W_s$ and $W_t - W_s$ are independent as defined in \eqref{BM Appendix}.
Observe that $W_s \sim N(0, s)$ and $W_t - W_s \sim N(0, t-s)$ via \eqref{BM Appendix}. 
Due to independence, we can write out the expected value as follows
\begin{align}
     \rE{S_0^2 e^{(r-\frac{1}{2}\sigma^2)(t+s)}e^{2 \sigma W_s} e^{\sigma (W_t - W_s)}}
     = S_0^2 e^{(r-\frac{1}{2}\sigma^2)(t+s)} \rE{e^{2 \sigma W_s}} \rE{e^{\sigma (W_t - W_s)}}
\end{align}
Using \eqref{moment 1} again, observe that
\begin{align}
     \rE{e^{2 \sigma W_s}} = e^{2 \sigma^2 s} \\
     \rE{e^{\sigma (W_t - W_s)}} = e^{\frac{1}{2} \sigma^2 (t-s)} \;
\end{align}
Substituting, we find that
\begin{align}
     \rE{S_{s} S_{t}} 
     = S_0^2 e^{(r-\frac{1}{2}\sigma^2)(t+s)} \rE{e^{2 \sigma W_s}} \rE{e^{\sigma (W_t - W_s)}}
     = S_0^2 e^{r(t+s)+\sigma^2s}
\end{align}
We now return to the main integral using this new result
\begin{align} \label{double int A_T^2}
     \frac{2}{T^2} \int_0^T \int_0^t S_0^2 e^{r(t+s)+\sigma^2s} \; ds \; dt = \frac{2}{T^2} \int_0^T S_0^2 e^{rt} \int_0^t e^{(r + \sigma^2)s} \; ds \; dt
\end{align}
We begin by evaluating the inside integral from \eqref{double int A_T^2}
\begin{align}
     \frac{2}{T^2} \int_0^T S_0^2 e^{rt} \int_0^t e^{(r + \sigma^2)s} \; ds \; dt = \frac{2}{T^2} \int_0^T \frac{S_0^2 e^{rt}}{r + \sigma^2} (e^{(r+\sigma^2)t} - 1) \; dt = \frac{2 S_0^2}{T^2(r + \sigma^2)} \int_0^T  (e^{(2r+\sigma^2)t} - e^{rt}) \; dt
\end{align}
Evaluating the next integral, we see that
\begin{align}
     \frac{2 S_0^2}{T^2(r + \sigma^2)} \int_0^T  (e^{(2r+\sigma^2)t} - e^{rt}) \; dt = \frac{2 S_0^2}{T^2(r + \sigma^2)} \left[ \frac{1}{2r+\sigma^2} \left( e^{(2r + \sigma^2)t} - 1 \right) - \frac{1}{r} \left( e^{rT} - 1 \right) \right]
\end{align}
Thus,
\begin{align}
     \rE{A_T^2} =  
     \frac{2 S_0^2}{T^2(r + \sigma^2)} \left[ \frac{1}{2r+\sigma^2} \left( e^{(2r + \sigma^2)t} - 1 \right) - \frac{1}{r} \left( e^{rT} - 1 \right) \right]
\end{align}
Continuing on to the RHS of \eqref{expect approx 2}, we can apply the moment generating function again with the information that $W_t \sim N(0, t)$ to get
\begin{align}
     \rE{Y_T^2}
     = Y_0^2 e^{(2r-\Gamma^2)T} \rE{e^{2\Gamma W_T}}
     = Y_0^2 e^{(2r+\Gamma^2)T}
\end{align}
Equating $\rE{A_T^2}$ and $\rE{Y_T^2}$ in \eqref{expect approx 2} and recalling the calibration \eqref{calibration 1 r neq 0} we found from \eqref{expect approx 1}, it follows that
\begin{align}
     \Gamma^2 = \frac{1}{T} \left( \ln{\left( \frac{2S_0^2}{Y_0^2 T^2 (r+\sigma^2)} \right)}+\ln{\left( \frac{1}{2r+\sigma^2} \left( e^{2rT+\sigma^2T} - 1 \right) -\frac{1}{r}(e^{rT}-1) \right)} -2rT \right)
\end{align}

\subsection{Approximating Arithmetic Asian under the Black-Scholes Model using the Bachelier Model}

If we assume $r=0$, stock price evolves as follows in the Bachelier Model
\begin{align}
     S_t = S_0 + \kappa W_t
\end{align}
One can price an option sold at time $t=0$ which expires at time $t=T$ according to
\begin{align}
     V_0 = e^{-rT} \rE{V_T} = \rE{V_T}
\end{align}
Therefore, we can price a European call as so
\begin{align}
     V_0 = \rE{(S_T-K)^{+}}
\end{align}
Note that $S_t \sim N(S_0,\kappa^2T)$, as the variance of a Brownian motion is defined as $T$ shown in \eqref{BM Appendix} and $\kappa$ is constant. 
Now we can price an Asian call as follows
\begin{align} \label{sect 6 asian}
     A_T = \frac{1}{T} \int_0^T S_t \; dt
\end{align}
The time 0 price is
\begin{align} \label{sect 6 t0 price}
     V_0 = \rE{(A_T-K)}^{+}
\end{align}
Expanding $S_t$ yields
\begin{align}
     \int_0^T S_t \; dt = \int_0^T S_0 \; dt + \int_0^T \kappa W_t \; dt = S_0T + \kappa \int_0^T W_t \; dt
\end{align}
Plugging this into \eqref{sect 6 asian} yields
\begin{align} 
     A_T = S_0 + \frac{\kappa}{T} \int_0^T W_t \; dt
\end{align}
Plugging this into \eqref{sect 6 t0 price} yields
\begin{align}
     V_0 = \rE{(A_T-K)}^{+} = \rE{({(S_0 + \frac{k}{T} \int_0^T W_t \; dt) - K})^+} 
\end{align}
Note that this is the same form as the European Call, except the European Call has $S_T$ inplace of $A_T$.
The variance of $A_T$ is $\frac{\kappa^2T}{3}$, as $\frac{\kappa}{T}$ is constant and the variance of $\int_0^T W_t \; dt$ is $\frac{T^3}{3}$ as shown in \eqref{BM Appendix}. %maybe add variance to appendix? fixme? 
\begin{align}
     \frac{\kappa^2}{T^2} \frac{T^3}{3}= \frac{\kappa^2 T}{3}
\end{align}
The variance of $A_T$ is exactly $\frac{1}{3}$ the variance of $S_T$, while the mean remains the same as $S_T$.
Variance is equal to volatility squared.
So, we can approximate the price of an Asian in the Black-Scholes Model by pricing it with the same parameters as a European Option 
but just changing the volatility to be $\frac{1}{\sqrt{3}}$ of the original. 
This result also applies to Asian Puts through Put-Call Parity\eqref{PC parity appendix}, 
using a European Put with $\frac{1}{\sqrt{3}}$ of the original volatility.

This is an approximation because the Black-Scholes and Bachelier Model scale differently.
(FIXME: Use data to show when this is an acceptable approximation)

The advantage of this approximation is that European Options are the most well-known option, so one could easily estimate
the price of an Asian Option using already existing infrastructure.


\subsection*{Acknowledgements} The authors would like to thank Prof. Hrusa for his patient guidance on this project.  


\appendix

\section{Notation and conventions}

For a random variable $X$ we use the notation $X^+$ to denote the random variable $\max(X,0)$. We note that by definition, we have 
\begin{align}\label{eq: pos part decomp}
      X = X^+ - (-X)^+,
\end{align}
from which the \emph{put-call parity} can be derived. 

For a normal random variable $X$ we use the notation $X \sim N(\mu,\sigma^2)$
to denote that it has mean $\mu$ and variance $\sigma^2$.

\section{Arbitrage-free pricing} \label{Arbitrage-Free Pricing Appendix}

Before defining arbitrage-free pricing, we first must define arbitrage. An arbitrage strategy has three properties. 
\begin{enumerate}
     \item the agent's initial capital is zero.
     \item the agent has zero percent chance of losing money.
     \item the agent has a strictly positive probability of profit. 
\end{enumerate}
Under this definition of arbitrage, the arbitrage-free price of an asset is the price where an arbitrage strategy is not possible.

\subsection{Arbitrage-free Market}

In this paper we work under the assumption that the market is arbitrage-free. As such, we claim that if the values of two portfolios are equal at time $T > 0$, then for all times $\tau$ where $0 \leq \tau \leq T$, the values of both portfolios are equal.

We prove by contrapositive. Assume that at time $T > 0$ the prices of two portfolios are equal, and that at time $\tau$ where $0 \leq \tau \leq T$ that one portfolio is worth more than the other. 
Let $P_1$ be the value of portfolio 1 and $P_2$ be the value of portfolio 2. Thus, without loss of generality, at time $\tau$, let $P_1 > P_2$.
At time $\tau$, we buy portfolio 2 and sell portfolio 1. We can pocket the difference $P_1 - P_2$. At time $T$, we can then sell portfolio 2 to pay off the time $T$ cost of portfolio 1. 
Thus, there exists an arbitrage strategy, which is a contradiction.

It follows that under an arbitrage-free model, if two portfolios have equal value at time $T$, they must have equal value at all times from $0$ to $T$.

\section{Put-Call Parity}\label{PC parity appendix}

An important result used repeatedly throughout this paper is put-call parity. 

For some asset of price $S_T$ at time $T$, define the European put and call of strike $K$ as 
\begin{align} 
     P^E_T = (K - S_T)^+ \\
     C^E_T = (S_T - K)^+.
\end{align}
By \eqref{eq: pos part decomp}, 
\begin{align} \label{Put-Call Parity European}
     P^E_T - C^E_T = K - S_T.
\end{align}
We can replicate the LHS portfolio by going long a put and short a call at time $\tau$. 
The RHS can be replicated by investing $Ke^{r(\tau - T)}$ into a risk-free return and shorting the asset at $T = \tau$.
Under the assumption of the arbitrage-free market, it follows that
\begin{align} 
     P^E_\tau - C^E_\tau = Ke^{r(\tau - T)} - S_\tau.
\end{align}
We apply a similar argument towards Asian puts and calls.
Again define an asset with price $S_T$ at time $T$.
Define the Asian put and call with strike price $K$ respectively as
\begin{align} 
     P^A_T = (K - \int_0^T S_t \;dt)^+ \\
     C^A_T = (\int_0^T S_t \;dt - K)^+.
\end{align}
By \eqref{eq: pos part decomp}, 
\begin{align} 
     P^A_T - C^A_T = K - \int_0^T S_t \;dt.
\end{align}
We replicate the LHS by going long the put and short the call at time $\tau$.
The RHS can be replicated by investing $Ke^{r(\tau - T)}$ and shorting an option at $w_\tau$ which pays $A_T$ at time $T$, all at time $\tau$.
Thus, 
\begin{align} \label{Put-Call Parity Asian}
     P^A_\tau - C^A_\tau = Ke^{r(\tau - T)} - w_\tau.
\end{align}
The calculation of $w_\tau$ is demonstrated in the above sections.

\section{Brownian Motion}\label{BM Appendix}
Brownian Motion is a stochastic process used to model evolution of asset prices in a continuous-time model.
The following properties are of use in this paper:
\begin{itemize}
     \item $W_0=0$
     \item The mapping of $t$ to $W_t$ is continuous
     \item For each $t \geq 0$, $W_t \sim N(0, t)$
     \item For all $s,t$ with $0 \leq s t$, we have $W_t-W_s \sim N(0, t-s)$
     \item For all $s_1, t_1, s_2, t_2$ with $0 \leq s_1 < t_1 \leq s_2 < t_2$ the variables $W_{t_1}-W_{s_1}$ and $W_{t_2}-W_{s_1}$ are independent
\end{itemize}
Another important property is that $\int^T_0 W_t \;dt \sim N(0, \frac{T^3}{3})$, which can be derived through stochastic calculus.
(FIXME: what tools) (FOLLOWUP: not a proper reference, but 

https://math.stackexchange.com/questions/1336471/variance-of-an-integral-of-brownian-motion 

shows a proof with integration by parts and Fubini),


\section{Max Transformation} \label{Max appendix}
We derive the following property of the max function:
\begin{align}
     \max(a, b) = a + \max(0, b-a).
\end{align}
The proof follows through casework. First consider when $a > b$, it follows that
\begin{align}
     \max(a, b) = a
\end{align}
\begin{align}
     a + \max(0, b-a) = a + 0 = a.
\end{align}
The second case we consider is $b \geq a$, then
\begin{align}
     \max(a, b) = b
\end{align}
\begin{align}
     a + \max(0, b-a) = a + (b - a) = b.
\end{align}
The proof is now complete.

\section{Moment Generating Functions}

We define $m_X(y)$ to be moment generating function on random variable $X$ such that 
\begin{align}
     m(y) = \rE{e^{yX}}
\end{align}
where $y \in \mathbb{R}$. When $X \sim N(\mu, \sigma^2)$, we have 
\begin{align} \label{mgf func}
     m(y) = \rE{e^{yX}} = e^{(\mu y + \frac{1}{2} \sigma^2 y^2)}
\end{align}
\section{Various replicating strategies}

\subsection{Replicating Puts and Calls}

The put and call options are two often used contracts which give the buyer the right, but not obligation, to respectively sell or buy an underlying security. 


To "price back" a put or call from time $t$ to time $0$, where $t \geq 0$, we simply buy a put or call respectively at time $0$. 
It thus follows that 
\begin{align} \label{replicating puts/calls}
     \rE{C_t} = C_0 \\
     \rE{P_t} = P_0
\end{align}
%\subsection{Replicating European options}
%\subsection{Replicating Asian options}

%\printbibliography


\end{document}