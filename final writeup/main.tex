\documentclass[reqno]{amsart}



\usepackage{amssymb,amsfonts,amsmath, amsthm}
\usepackage{lipsum} 
\usepackage[all,arc]{xy}
\usepackage{enumerate}
\usepackage{mathrsfs}
\usepackage[usenames, dvipsnames]{xcolor}
\definecolor{cerulean}{RGB}{0,123,167}
\usepackage[colorlinks = true, linkcolor = cerulean, citecolor = Thistle]{hyperref}
\usepackage{tikz}
\usepackage{tikz-cd}
\usepackage[super]{nth}
\usetikzlibrary{arrows}
\usepackage{caption}
\usepackage{geometry}
\geometry{margin = 0.8in, tmargin=0.8in}
\usepackage{multicol}
\usepackage{esint}
\usepackage{mathtools}
\usepackage{import}
\usepackage{xifthen}
\usepackage{pdfpages}
\usepackage{transparent}
\usepackage{upgreek}
\usepackage{pgfplots}
\pgfplotsset{compat=1.17} 
\usepackage{bbm}

\newcommand{\incfig}[1]{
	\fontsize{9pt}{12pt}\selectfont
    \def\svgwidth{\columnwidth}
    \resizebox{0.8\textwidth}{!}{\import{./img/}{#1.pdf_tex}}
}
\newcommand{\C}{\mathbb{C}}
\newcommand{\E}{\mathbb{E}}
\newcommand{\F}{\mathbb{F}}	
\newcommand{\T}{\mathbb{T}}
\newcommand{\R}{\mathbb{R}}
\newcommand{\Q}{\mathbb{Q}}
\newcommand{\N}{\mathbb{N}}
\newcommand{\Z}{\mathbb{Z}}
\newcommand{\K}{\mathbb{K}}
\newcommand{\bH}{\mathbb{H}}
\newcommand{\bP}{\mathbb{P}}
\newcommand{\bS}{\mathbb{S}}
\newcommand{\cO}{\mathcal{O}}
\newcommand{\cU}{\mathcal{U}}
\newcommand{\cV}{\mathcal{V}}
\newcommand{\U}{\mathcal{U}}
\newcommand{\ds}{\displaystyle}
\newcommand{\intbar}{\fint}
\newcommand{\ve}{\varepsilon}
\newcommand{\p}{\partial}
\newcommand{\bvec}{\boldsymbol}
\newcommand{\abs}[1]{\vert#1\vert}
\newcommand{\nab}{\nabla}
\providecommand{\norm}[1]{\left\Vert#1\right\Vert}
\newcommand{\weakar}{\rightharpoonup} 
\newcommand{\weakstarar}{\overset{\ast}{\rightharpoonup}}
\newcommand{\triplenorm}[1]{{\left\vert\kern-0.25ex\left\vert\kern-0.25ex\left\vert #1 
    \right\vert\kern-0.25ex\right\vert\kern-0.25ex\right\vert}}
\newenvironment{sol}
  {\begin{proof}[Solution]}
  {\end{proof}}
\newcommand{\paren}[1]{\left(#1\right)}
\newcommand{\Hzero}{{}_0H^1}
\newcommand{\Hzeros}[1]{{}_0H^{#1}}
\newcommand{\Hzerosig}{{}_0H^1_\sigma}
\renewcommand{\Re}{\operatorname{Re}}
\renewcommand{\Im}{\operatorname{Im}}
\newcommand{\leb}{\mathcal{L}_o^N}
\newcommand{\hauss}{\mathcal{H}_o^1}
\numberwithin{equation}{section}
\newcommand{\lap}[1]{\mathcal{L} \left\{ #1 \right\}}
\newcommand{\lapinv}[1]{\mathcal{L}^{-1}\left\{#1\right\}} 

\DeclareMathOperator{\Span}{span}
\DeclareMathOperator{\esssup}{esssup}
\DeclareMathOperator{\supp}{supp}
\DeclareMathOperator{\curl}{curl}
\DeclareMathOperator{\sgn}{sgn}
\DeclareMathOperator{\lip}{Lip}
\DeclareMathOperator{\epi}{epi}
\DeclareMathOperator{\gr}{Gr}
\DeclareMathOperator{\diam}{diam}
\DeclareMathOperator{\dist}{dist}
\DeclareMathOperator{\meas}{meas}
\DeclareMathOperator{\tr}{tr}
\DeclareMathOperator{\Tr}{Tr}
\DeclareMathOperator{\diverge}{div}
\DeclareMathOperator{\rank}{rank}
\DeclareMathOperator{\proj}{proj}
\DeclareMathOperator{\Range}{Ran}
\DeclareMathOperator{\sym}{sym}


\newtheorem{thm}{Theorem}[section]
\newtheorem{cor}[thm]{Corollary}
\newtheorem{prop}[thm]{Proposition}
\newtheorem{lem}[thm]{Lemma}
\newtheorem{conj}[thm]{Conjecture}
\newtheorem{quest}[thm]{Question}
\theoremstyle{definition}
\newtheorem*{thm*}{Theorem}
\newtheorem*{def*}{Definition}
\newtheorem*{prop*}{Proposition}
\newtheorem{defn}[thm]{Definition}
\newtheorem{defns}[thm]{Definitions}
\newtheorem{nota}[thm]{Notation}
\newtheorem{con}[thm]{Construction}
\newtheorem{exmp}[thm]{Example}
\newtheorem{exmps}[thm]{Examples}
\newtheorem{notn}[thm]{Notation}
\newtheorem{notns}[thm]{Notations}
\newtheorem{addm}[thm]{Addendum}
\newtheorem{exer}[thm]{Exercise}
\newtheorem{prob}[thm]{Problem}
\theoremstyle{remark}
\newtheorem{rem}[thm]{Remark}
\newtheorem{rems}[thm]{Remarks}
\newtheorem{warn}[thm]{Warning}
\newtheorem{sch}[thm]{Scholium}


\newenvironment{nouppercase}{
  \let\uppercase\relax
  \renewcommand{\uppercasenonmath}[1]{}}{}
\newcommand{\rvline}{\hspace*{-\arraycolsep}\vline\hspace*{-\arraycolsep}}
\renewcommand{\P}{\mathbb{P}}
\newcommand{\D}{\mathcal{D}}
\newcommand{\imp}{\;\Rightarrow\;}
\newcommand{\m}{\mathrm}
\newcommand{\opl}{\oplus}
\newcommand{\ot}{\otimes}
\newcommand{\lv}{\lVert}
\newcommand{\rv}{\rVert}
\newcommand{\Lm}{\Lambda}
\newcommand{\J}{\boldsymbol{J}}
\newcommand{\lm}{\lambda}
\newcommand{\al}{\alpha}
\newcommand{\be}{\beta}
\newcommand{\es}{\varnothing}
\newcommand{\lra}{\;\Leftrightarrow\;}
\newcommand{\ep}{\varepsilon}
\newcommand{\f}{\frac}
\newcommand{\sig}{\sigma}
\newcommand{\gam}{\gamma}
\newcommand{\del}{\delta}
\newcommand{\thet}{\vartheta}
\newcommand{\bn}{\binom}
\newcommand{\oo}{^\circ}
\newcommand{\pd}{\partial}
\newcommand{\lf}{\lfloor}
\newcommand{\rf}{\rfloor}
\newcommand{\grad}{\nabla}
\newcommand{\bpm}{\begin{pmatrix}}
\newcommand{\epm}{\end{pmatrix}}
\newcommand{\para}{\parallel}
\newcommand{\loc}{\m{loc}}
\renewcommand{\bar}{\overline}
\newcommand{\Le}{\mathfrak{L}}
\newcommand{\Leo}{\mathfrak{L}_{\m{o}}}
\newcommand{\emb}{\hookrightarrow}
\newcommand{\res}{\restriction}
\renewcommand{\le}{\leqslant}
\renewcommand{\ge}{\geqslant}
\newcommand{\jump}[1]{\left\llbracket#1\right\rrbracket}
\newcommand{\tjump}[1]{\llbracket#1\rrbracket}
\newcommand{\sjump}[1]{\big\llbracket#1\big\rrbracket}
\newcommand{\bjump}[1]{\Big\llbracket#1\Big\rrbracket}
\newcommand{\bnorm}[1]{\Big\lv#1\Big\rv}
\newcommand{\snorm}[1]{\big\lv#1\big\rv}
\newcommand{\tnorm}[1]{\lv#1\rv}
\newcommand{\bp}[1]{\Big(#1\Big)}
\renewcommand{\sp}[1]{\big(#1\big)}
\newcommand{\tp}[1]{(#1)}
\newcommand{\babs}[1]{\Big|#1\Big|}
\newcommand{\sabs}[1]{\big|#1\big|}
\newcommand{\tabs}[1]{|#1|}
\renewcommand{\sb}[1]{\left[{#1}\right]}
\newcommand{\bsb}[1]{\Big[{#1}\Big]}
\newcommand{\ssb}[1]{\big[{#1}\big]}
\newcommand{\tsb}[1]{[{#1}]}
\newcommand{\cb}[1]{\left\{{#1}\right\}}
\newcommand{\scb}[1]{\big\{{#1}\big\}}
\newcommand{\bcb}[1]{\Big\{{#1}\Big\}}
\newcommand{\tcb}[1]{\{{#1}\}}
\newcommand{\br}[1]{\left\langle #1 \right\rangle}
\providecommand{\bbr}[1]{\Big\langle #1 \Big\rangle}
\providecommand{\sbr}[1]{\big\langle #1 \big\rangle}
\providecommand{\tbr}[1]{\langle #1 \rangle}
\newcommand{\ssum}[2]{{\textstyle\sum\limits_{#1}^{#2}}}
\newcommand{\z}[1]{\mathring{#1}}
\newcommand{\ip}[2]{\p{{#1},{#2}}}
\renewcommand{\bf}[1]{\mathbf{#1}}
\newcommand{\ii}{\m{i}}
\DeclareMathOperator{\Div}{div}
\DeclareMathOperator{\ten}{ten}
\DeclareMathOperator{\card}{card}
\pdfstringdefDisableCommands{\def\eqref#1{(\ref{#1})}}



\title{MFSURP}

\author{Jessica Chen}
\address{
Carnegie Mellon University\\
Pittsburgh, PA 15213, USA
}
\email[J. Chen]{jschen2@andrew.cmu.edu}


\author{Linxuan Jiang}
\address{
Carnegie Mellon University\\
Pittsburgh, PA 15213, USA
}
\email[L. Jiang]{linxuanj@andrew.cmu.edu}

\author{Frank Sacco}
\address{
Carnegie Mellon University\\
Pittsburgh, PA 15213, USA
}
\email[F. Sacco]{fsacco@andrew.cmu.edu}

\author{Albert Zhang}
\address{
Carnegie Mellon University\\
Pittsburgh, PA 15213, USA
}
\email[A. Zhang]{albertzh@andrew.cmu.edu}

\thanks{J. Chen, L. Jiang, F. Sacco, A. Zhang were supported by the MFSURP program at Carnegie Mellon University.}

\keywords{Blachelier model, Chooser options, Exotic options.}


\begin{document}


\begin{abstract}
    In this set of notes we derive the time-zero prices of various \emph{chooser options}. These are contracts with a fixed maturity date $T$ and a chooser date $\tau$ satisfying $0 \le \tau \le T$, for which an agent is allowed to choose at time $\tau$ the underlying security that determines the structure of the payoff at time $T$.  
\end{abstract}



\maketitle  
\tableofcontents

\section{Introduction}

\subsection{Arbitrage-free pricing}

(FIXME: add some background information on arbitrage-free pricing, aka summarize what you learned in 21-270. We can also add this to an appendix.)
Before defining arbitrage-free pricing, we first must define arbitrage. An arbitrage strategy has three properties. 
\begin{enumerate}
     \item the agent's initial capital is zero.
     \item the agent has zero percent chance of losing money.
     \item the agent has a strictly positive probability of profit. 
\end{enumerate}
Under this definition of arbitrage, the arbitrage-free price of an asset is the price where an arbitrage strategy is not possible.

\subsection{Arbitrage-free Market}

In this paper we work under the assumption that the market is arbitrage free. As such, we claim that if the values of two portfolios are equal at time $T > 0$, then for all times $tau$ where $0 \leq \tau \leq T$, the values of both portfolios are equal.

We prove by contrapositive. Assume that at time $T > 0$ the prices of two portfolios are equal, and that at time $\tau$ where $0 \leq \tau \leq T$ that one portfolio is worth more than the other. 
Let $P_1$ be the value of portfolio 1 and $P_2$ be the value of portfolio 2. WLOG, at time $\tau$, let $P_1 > P_2$.
At time $tau$, we buy portfolio 2 and sell portfolio 1. We can pocket the difference $P_1 - P_2$. At time $T$, we can then sell portfolio 2 and use that money to pay off what we owe from portfolio 1. 
Thus, there exists an arbitrage strategy, a strategy with zero initial capital and a strictly positive probability of profit.

By contrapositive, it thus follows that under an arbitrage-free model, if two portfolios have equal value at time $T$, they must have equal value at all times from $0$ to $T$.

\subsection{The Blachelier Model}
In this paper we work within the context of the \emph{Blachelier model}, where the stock prices $\{S_t\}_{t \ge 0}$ evolves according to 
\begin{align}\label{eq: r not 0}
	 S_t = e^{rt} \left( S_0 + \kappa^{-rt}W_t + \kappa r \int_0^t e^{-rs} W_s \; ds \right),
\end{align}
where $S_0 > 0$ denotes the initial stock price at time 0 and $\{W_t\}_{t \ge 0}$ is a Brownian motion under the risk neutral measure $\tilde{\mathbb{P}}$. We note to the reader that in the special case when $r = 0$, \eqref{eq: r not 0} reduces to 
\begin{align}\label{eq: r=0}
      S_t = S_0 + \kappa W_t. 
\end{align}

(FIXME: explain to the reader why the Blachlier model is interesting).
Recently, in April 2020, oil prices finished in the negatives. The often used Black-Scholes model, however, is unable to handle negative prices, which sparked interest in the Bachelier model, a mathematically similar model with the advantage of being able to handle negative prices.


\subsection{Notation and conventions}

For a random variable $X$ we use the notation $X^+$ to denote the random variable $\max(X,0)$. We note that by definition, we have 
\begin{align}\label{eq: pos part decomp}
      X = X^+ - (-X)^+,
\end{align}
from which the \emph{put-call parity} can be derived. 

(FIXME: some common notation that needs to be explained: 
\begin{enumerate}
      \item the notation $X \sim N(\mu,\sigma^2)$
      \item $\ldots$
\end{enumerate}
)

\section{Arbitrage-free pricing under the Blachelier model}
In this section we derive the arbitrage-free prices of some contracts in the simplest setting when $r = 0$.  

\subsubsection{European Call}
We first consider a European call where the payoff at time $T$ is given by 
\begin{align}
	 V_T = (S_T - K)^+
\end{align}
for a fixed strike price $K$. We note that under $\tilde{\mathbb{P}}$, $ W_T \sim N(0, T)$, therefore
\begin{align}\label{eq: S_T dist}
S_T \sim N(S_0, \kappa^2T) \; \text{under the risk neutral measure} \; \tilde{\mathbb{P}}.
\end{align}
According to the risk neutral pricing formula, the time-0 price of this security is given by 
\begin{align}
	 V_0 = \tilde{\E}[(S_T - K)^+].
\end{align}
Recall that if we have a random variable $X$ with probability density function $f_X$ under a probability measure $\mathbb{P}$, then the ``law of the unconscious statistician'' tells us that 
\begin{align}
	 \E[g(X)] = \int_{-\infty}^\infty g(x) f_X(x) \; dx.
\end{align}
In our setting, we have 
\begin{align}
	 g(S_T) = (S_T - K)^+,
\end{align}
and the distribution of $S_T$ under $\tilde{\bP}$ as a random variable is given in \eqref{eq: S_T dist}. Therefore, the time-zero price $V_0$ is given by 
\begin{align}\label{eq: call int}
	 V_0 = \tilde{\E}[(S_T - K)^+] =  \tilde{\E}[g(S_T)] = \int_{-\infty}^\infty (x-K)^+ \psi(x) \; dx,
\end{align}
where 
\begin{align}
	 \psi(x) = \frac{1}{\nu}\varphi\left(\frac{x-\mu}{\nu}\right), \; \varphi(y) = \frac{1}{\sqrt{2\pi}}e^{-y^2/2}, 
\end{align}
and
\begin{align}
	 \mu = S_0, \; \nu = \kappa \sqrt{T}.
\end{align}
To compute \eqref{eq: call int}, we first note that since $(x-K)^+ = 0$ for $x \le K$, the domain of integration is the set $\{x \mid x \ge K\}$. Now we use the change of variables 
\begin{align}
	 y = -\frac{x-\mu}{\nu} \Longleftrightarrow x = \mu - \nu y,
\end{align}
and we note that since $\nu > 0$, 
\begin{align}
	 x \ge K \Longleftrightarrow \frac{x-\mu}{\nu} \ge \frac{K - \mu}{\nu} \Longleftrightarrow y \le \frac{\mu - K}{\nu} =: d_-.
\end{align}
Then by performing a change of variables, \eqref{eq: call int} becomes 
\begin{align}
	 V_0 = \int_{-\infty}^{d_-} (\nu y + K - \mu) \varphi(-y) \; dy = \int_{-\infty}^{d_-} (\nu y + K - \mu) \varphi(y) \; dy = \underbrace{\int_{-\infty}^{d_-} \nu y \varphi(y) \; dy}_{:= I} + \underbrace{\int_{-\infty}^{d_-}  (K-\mu)\varphi(y) \; dy}_{:= II}.
\end{align}
We define the cumulative distribution function of a standard normal random variable $X$ under $\mathbb{P}$ via 
\begin{align}
	 \varphi(x) = \mathbb{P}[X \le x] = \E[ \mathbbm{1}_{X \le x}] = \int_{-\infty}^x \varphi(y) \; dy.
\end{align}
With this notation in hand, we can write 
\begin{align}
	 II = (K-\mu) \int_{-\infty}^{d_-} \varphi(y) \; dy = (K-\mu) \varphi(d_-),
\end{align}
and 
\begin{align}
	 I = \nu \int_{-\infty}^{d_-} y \varphi(y) \; dy =  \frac{\nu}{\sqrt{2\pi}} \lim_{t \to -\infty} (e^{-t^2/2} - e^{-d_-^2/2}) = -\frac{\nu}{\sqrt{2\pi}} e^{-d_-^2/2}.
\end{align}
Therefore 
\begin{align}
	 V_0 =  -\frac{\nu}{\sqrt{2\pi}} e^{-d_-^2/2} + (K-\mu) \varphi(d_-).
\end{align}
To compute the price of a put, one can use put-call parity (FIXME: derive put-call parity somewhere and reference it). 
\subsubsection{Arithmetic Asian calls}
Next we consider an \emph{arithmetic Asian call} where the payoff at time $T$ is given by 
\begin{align} \label{eq: asian}
	 V_T = (A_T - K)^T, \; A_T = \frac{1}{T}\int_0^T S_t \; dt = S_0 + \frac{\kappa}{T} \int_0^T W_t \; dt. 
\end{align}
Using tools from Stochastic calculus, one can show that under the risk neutral measure $\tilde{\mathbb{P}}$, 
\begin{align}
	 \int_0^T W_t \; dt \sim N(0, T^3/3).
\end{align}
Therefore we have 
\begin{align}
	 A_T \sim N(S_0, \kappa^2 T/3) \; \text{under the risk neutral measure} \; \tilde{\mathbb{P}}.
\end{align}
Comparing this to \eqref{eq: S_T dist}, we see that $A_T$ has a similar distribution, the only difference is that the variance of $A_T$ is smaller by a factor of $1/3$, so the standard deviation of $A_T$ is smaller by a factor of $1/\sqrt{3}$. By performing the exact same set of calculation as before, the time-0 price of an Asian option is 
\begin{align} \label{eq: asian op}
	 V_0 = -\frac{\nu}{\sqrt{3}\sqrt{2\pi}} e^{-3d_-^2/2} + (K-\mu) \varphi(\sqrt{3}d_-),
\end{align}
where 
\begin{align}
	 \mu = S_0, \; \nu = \kappa \sqrt{T}, \; d_- = \frac{\mu - K}{\nu}.
\end{align}
We note that since $\sqrt{3} > 1$, we see from \eqref{eq: asian op} that the price of an Asian option is higher than the price of a European call. This should be expected as one is paying a premium for a less volatile product. 

\section{Chooser options}
In this section we consider a more complicated type of financial contracts known as \emph{chooser options}. These are contracts with a fixed maturity date $T$ and a strike price $K$, and an agent is allowed to decide on a choosing date $\tau < T$ to choose the underlying derivative in the contract. Many results are known when an agent is allowed to choose between a European call and a European put; we are interested in a variant of this type of contract that allows an agent to decide between two securities that pays 
\begin{align}
     C_T = (A_T - K)^+, \; P_T = (K - A_T)^+,
\end{align}
where $A_T$ is defined via \eqref{eq: asian}.
Here, we assume the agent chooses optimally with no outside information. At time $\tau$, the agent will choose the option of higher value between the put and call, therefore the value of this contract at time $\tau$ is 
\begin{align}\label{eq: V_tau}
     V_\tau = \max(C_\tau, P_\tau).
\end{align}
The time-zero price of this contract is then 
\begin{align}
      V_0 = \tilde{\E}[V_\tau].
\end{align}
In the next subsection, we simplify the expression for $V_\tau$ via the method of replication. 
\subsection{Replication}
We first note that by properties of the max function (FIXME: maybe add this in intro and reference?), we can write 
\begin{align}
     V_\tau = C_\tau + \max(0, P_\tau - C_\tau).
\end{align}
By \eqref{eq: pos part decomp}, we have
\begin{align}\label{eq: P - C}
     P_T - C_T = (K-S_T)^+ - (S_T - K)^+ = K - A_T. 
\end{align}
Combining \eqref{eq: V_tau} and \eqref{eq: P - C} then gives us 
\begin{align}
      V_\tau = C_\tau + \max(0, K - A_T).
\end{align}
Next, we identify the time-$\tau$ prices of contracts paying $P_T - C_T$ and $K - A_T$ at time $T$.

To replicate a security with payoff $P_T - C_T$, we consider a portfolio that longs a put and shorts a call at time $0$, both with maturity $T$ and strike $K$ (FIXME: what's the maturity?). To replicate a security with payoff $K - A_T$,  we consider a portfolio investing $Ke^{-rT}$ into the money account at time 0 and shorting a contract (FIXME: what kind of contract?) (FOLLOWUP I'm tempted to say ZCB but I'm not sure) to receive $A_T$ at time $T$. 

Since both portfolios have the same payoff at time $T$ by \eqref{eq: P - C}, they have the same price for all times $t$ where $0 \leq t \leq T$ (FIXME: in the intro explain why this is true in terms of arbitrage). (FOLLOWUP added to section 1.2 in intro, idk how to ref)

For any $t$ satisfying $0 \le t \le T$, we define $P_t$ to be the time-$t$ value of a put with payoff $P_T$ at time $T$, $C_t$ to be the time-$t$ value of a call with payoff $C_T$ at time $t$, $A_t$ to be the time-$t$ value of an asian option paying $A_T$ at time $t$. (FIXME: maybe move to intro?)

Using this notation, at time $\tau$ the value of the first portfolio is $P_\tau - C_\tau$. Also, at time $\tau$ the second portfolio has $Ke^{-rT + r\tau}$ in the bank and is shorting a contract which pays $A_T$ at $T$, therefore the time-$\tau$ value of the second portfolio is $Ke^{r(\tau - T)} - A_\tau$. We denote the value of a contract at time $\tau$ which pays $A_T$ at time $T$ as $w_\tau$.
By replication, the time $\tau$ prices of the portfolios are equal, therefore we have 
\begin{align}\label{eq: P-C Parity}
     P_\tau - C_\tau = Ke^{r(\tau - T)} - w_\tau.
\end{align}
Substituting this result back into \eqref{eq: V_tau}, the value of the original chooser contract at $\tau$ is
\begin{align}
     V_\tau = C_\tau + \max(0, Ke^{r(\tau - T)} - w_\tau).
\end{align}
Our next goal is to find an explicit formula for $w_\tau$. 

For simplicity, we define $U_\tau$ to be the time $\tau$ price of a contract with payoff $Y_T$ at time $T$, where $Y_T$ is defined via
\begin{align}\label{eq: y}
     Y_T = \int_0^T S_t \; dt.
\end{align}
Once $U_\tau$ is determined, then we can recover $w_\tau$ as $w_\tau = \dfrac{U_\tau}{T}$. 

Note that \eqref{eq: y} can be split into
two parts, 
\begin{align}
     Y_T = \int_0^\tau S_t \; dt + \int_\tau^T S_t \; dt.
\end{align}
Observe that the integral from $0$ to $\tau$ is known at time $\tau$ as each price $S_t$ will be known by the time $\tau$. So we can treat this integral as a constant and now try to replicate the integral from time $\tau$ to $T$.


\subsection{Replicating Asian options when $r > 0$}
We begin our replicating strategy by buying $x$ shares of stock at time $\tau$. For all times $t$ where $\tau \leq t \leq T$, we will continuously sell off stock at the rate $\alpha_t$ and invest the revenue.
With this strategy, at time $T$, the bank has 
\begin{align}
     \int_\tau^T \alpha_t S_t e^{r(T-t)} \; dt
\end{align}
To finish the replication, we want our replicating portfolio to be equal to the integral we are replicating:
\begin{align}
     \int_\tau^T \alpha_t S_t e^{r(T-t)} \; dt = \int_\tau^T S_t \; dt.
\end{align}
Solving for $\alpha_t$, we find that
\begin{align}
     \alpha_t = e^{r(t-T)}
\end{align}
Thus, the amount of shares our strategy started with was
\begin{align}
     x = \int_\tau^T e^{r(t-T)} \; dt = \dfrac{1}{r} - \dfrac{e^{r(\tau - T)}}{r}.
\end{align}
This tells us that the cost at time $\tau$ to receive the stock from times $\tau$ to $T$ continuously is $x S_\tau$. This gives us 
\begin{align}
     U_\tau = \int_0^\tau S_t \; dt + \dfrac{S_\tau}{r}\left( 1 - e^{r(\tau - T)} \right)
\end{align}
Recall that $w_\tau$ is the price at time $\tau$ to receive $A_T$, equivalent to $\dfrac{Y_T}{T}$, at time $T$.
Thus, the price at $\tau$ to receive just $A_T$ is 
\begin{align}
     w_\tau = \dfrac{U_\tau}{T} = \dfrac{\int_0^\tau S_t \; dt + \dfrac{S_\tau}{r}\left( 1 - e^{r(\tau - T)} \right)}{T}
\end{align}
Returning to \eqref{eq: P-C Parity} (FIXME: fix reference), we can write out the equation as
\begin{align}
     P_\tau - C_\tau = Ke^{r(\tau - T)} - \dfrac{\int_0^\tau S_t \; dt + \dfrac{S_\tau}{r}\left( 1 - e^{r(\tau - T)} \right)}{T}.
\end{align}
Substituting this into the chooser option from 1.7, the value of $V_\tau$ is
\begin{align}
     V_\tau = C_\tau + \max(0, Ke^{r(\tau - T)} - \dfrac{\int_0^\tau S_t \; dt + \dfrac{S_\tau}{r}\left( 1 - e^{r(\tau - T)} \right)}{T})
\end{align}


\subsection{Replicating Asian options when $r = 0$}
The above formula breaks when $r = 0$ since we divide by $r$ (FIXME: this is a little confusing since the interest rate was never specified. maybe specify that you are assuming $r > 0$ initially and now you're considering $r = 0$ as a special case. If you choose to do this I recommend breaking this section into different subsections). To fix this, we return to our replicating strategy for $w_\tau$ accounting for this special case.

Define $U_\tau$ and $Y_T$ the same way as above. Again, split the integral $Y_T$ such that
\begin{align}
     Y_T = \int_0^\tau S_t \; dt + \int_\tau^T S_t \; dt
\end{align}
We now replicate the integral from time $\tau$ to $T$ for the special case. We follow the same replicating strategy as before. Purchase $x$ shares of stock. For all times $t$ where $\tau \leq t \leq T$, we continuously sell off at the rate $\alpha_t$ and invest the revenue. 
By time $T$, the bank will have 
\begin{align}
     \int_\tau^T \alpha_t S_t \; dt
\end{align}
We finish the replication by setting this equal to the value we're replicating
\begin{align}
     \int_\tau^T \alpha_t S_t \; dt = \int_\tau^T S_t \; dt
\end{align}
Solving for $\alpha_t$, we see that when $r = 0$ that $\alpha_t = 1$. Thus, the number of shares the strategy started with was
\begin{align}
     \int_\tau^T dt = T - \tau
\end{align}
Similar to the $r \neq 0$ case, it then follows that
\begin{align}
     U_\tau = \int_0^\tau S_t dt + S_\tau (T - \tau), \; w_\tau = \dfrac{U_\tau}{T} = \dfrac{\int_0^\tau S_t dt + S_\tau (T - \tau)}{T}
\end{align}
Thus, Put-Call Parity in the special case tells us that 
\begin{align}
     P_\tau - C_\tau = K - \dfrac{U_\tau}{T} = \dfrac{\int_0^\tau S_t dt + S_\tau (T - \tau)}{T}.
\end{align}
Substituting this result into the chooser option formula, we have
\begin{align}
     V_\tau = C_\tau + \max(0, K - \dfrac{\int_0^\tau S_t \; dt + S_\tau (T - \tau)}{T})
\end{align}
Note that when the interest rate is $0$, the stock prices evolve according to 
\begin{align}
     S_t = S_0 + \kappa W_t
\end{align}
where $S_0 > 0$ and $\{W_t\}_{t \ge 0}$ is a Brownian motion under the risk neutral measure. We can now rewrite our chooser option formula as
\begin{align}
     V_\tau = C_\tau + \max(0, K - \dfrac{\int_0^\tau \left( S_0 + \kappa W_t \right) \; dt + (S_0 + \kappa W_\tau) (T - \tau)}{T})
\end{align}
So in conclusion, we find that 
\begin{align}
     V_\tau = C_\tau + \left(K- S_0 - \frac{\kappa(T-\tau)}{T} W_\tau - \frac{\kappa}{T} \int_0^\tau W_t \; dt  \right)^+.
\end{align}
Then by the risk-neutral pricing formula and the linearity of expection, the time-zero price $V_0$ is given by
\begin{align} \label{eq: expect form}
     V_0 = \tilde{\E}[C_\tau] + \tilde{\E}\left[\left(K- S_0 - \frac{\kappa(T-\tau)}{T} W_\tau - \frac{\kappa}{T} \int_0^\tau W_t \; dt  \right)^+\right].
\end{align}
Let $X$ be the random variable defined via 
\begin{align}
     X &= \frac{\kappa(T-\tau)}{T} W_\tau + \frac{\kappa}{T} \int_0^\tau W_t \; dt.
\end{align}
We now calculate the mean and variance of random variable $X$. We define $X$ as the sum of random variables
\begin{align}
     Y &= \frac{\kappa(T-\tau)}{T} W_\tau\\
     Z &= \frac{\kappa}{T} \int_0^\tau W_t \; dt.
\end{align}
Recall from the intro (add details on brownian motion in intro later) that the mean of the brownian motions in both $Y$ and $Z$ are $0$, thus the means of both $Y$ and $Z$ are 0. 
We now calculate the variance of $X$ as the sum of two random variables
\begin{align}
     Var(X) = Var(Y + Z)
\end{align}
It is known that
\begin{align}
     Var(Y + Z) = Var(Y) + Var(Z) + 2Cov(YZ).
\end{align}
Recall from the intro (add later) the variances of brownian motion. It follows that 
\begin{align}
     Var(Y) &= \tau(\frac{\kappa(T-\tau)}{T})^2\\
     Var(Z) &= \frac{\tau^3}{3}(\frac{\kappa}{T})^2
\end{align}
To calculate the covariance term, we expand it out in terms of expected value. Recall that the expected values of the brownian motions are 0.
\begin{align}
     Cov(XY) = \E(XY) - \E(X)\E(Y) = \E(XY)
\end{align}
We can now rewrite the covariance as
\begin{align}
     Cov(XY) = \E(W_\tau \int_0^\tau W_t \; dt) \;\dfrac{\kappa^2 (T - \tau)}{T^2}
\end{align}
For simplicity, let $\alpha = \dfrac{\kappa^2 (T - \tau)}{T^2}$. By a property of integrals and expected value, we can move the integral outside the expected value as such (probably need to fix this).
\begin{align}
     \alpha \E(W_\tau \int_0^\tau W_t \; dt) = \alpha \int_0^\tau \E(w_\tau w_t) dt
\end{align}
Observe that $t \leq \tau$. Thus, we can further simplify down to
\begin{align}
     \alpha \int_0^\tau \E((w_\tau + w_t - w_t) w_t) dt = \alpha \int_0^\tau \E(w_t^2 + (w_\tau - w_t) w_t) dt
\end{align}
We can expand the expected value by linearity of expectations. Recall (another brownian motion thing for intro ig) that the expected value of $(w_\tau - w_t) w_t$ is 0 and that the expected value of $w_t^2$ is $t$. Thus, we have
\begin{align}
     Cov(XY) = \alpha \int_0^\tau t \; dt = \alpha \dfrac{\tau^2}{2}
\end{align}

It follows that the mean and variance of $X$ can be computed as (IMPORTANT FIXME: add details of this computation!)
\begin{align}
     \mu &= \E(X) = 0\\
     \sigma^2 &= Var(X) = \tau(\frac{\kappa(T-\tau)}{T})^2 + \frac{\tau^3}{3}(\frac{\kappa}{T})^2 + \tau^2\frac{\kappa^2(T-\tau)}{T^2}.\\
     \nu &= \sigma
\end{align}
(FIXME: typically $\sigma$ denotes the standard deviation, so this isn't very consistent with standard notation)

Using methods of stochastic calculus (FIXME: we need to find a reference to cite), $X$ is normally distributed with mean $\mu$ and variance $\sigma^2$. Therefore we can define the probability density function as
\begin{align}
     \label{eq: phi-PDF}
     \varphi(x) &=\frac{1}{\sqrt{2\pi}}e^{-\frac{x^2}{2}}\\
     \psi(x) &= \frac{1}{\nu}\varphi(\frac{x-\mu}{\nu})
\end{align}
Now we can substitute back into our equation from \eqref{eq: expect form} and use the definition of expectation on a continuous random variable to get 
\begin{align} \label{eq: prechange}
     V_0 = \tilde{\E}[C_\tau] + \int_{-\infty}^{\infty}\left(K- S_0 - x\right)^+ \psi(x) \; dx.
\end{align}
To integrate the second term in $V_0$ we will let
\begin{align}
     z &= \frac{x-\mu}{\sigma}.
\end{align}
it follows that 
\begin{align}\label{eq: change variable}
     x &= z\nu + \mu\\
     dx &= \nu dz.
\end{align}
We note that 
\begin{align} \label{eq: int domain}
    K - S_0 - x \ge 0 \Longleftrightarrow x \leq K- S_0 \Longleftrightarrow  \frac{x-\mu}{\sigma} \leq \frac{K-S_0-\mu}{\nu}.
\end{align}
We define $d_-$ via
\begin{align}
      d_- =  \frac{K-S_0-\mu}{\nu}
\end{align}
so by \eqref{eq: int domain}, we have 
\begin{align} \label{eq: change bound}
     K - S_0 - x \ge 0  \Longleftrightarrow z \leq d_.
\end{align}
Now using \eqref{eq: change variable} and \eqref{eq: change bound} we can rearrange \eqref{eq: prechange} as:
\begin{align}
     V_0 = \tilde{\E}[C_\tau] + \left(\int_{-\infty}^{d\_}\left(K- S_0 - z\nu - \mu\right)\varphi(z) \; dz\right).
\end{align}
Simplifying this expression we get the form
\begin{align}
     V_0 = \tilde{\E}[C_\tau] - \nu \int_{-\infty}^{d_-}z\varphi(z)\nu \; dz +  (K- S_0 - \mu)\int_{-\infty}^{d_-}\varphi(z) \; dz 
\end{align}
Excluding the call, we can easily simplify the latter of the terms by defining the cumulative distribution function
\begin{align}
     \Phi(x) = \int_{-\infty}^x \varphi(y) dy.
\end{align}
We resolve the former term by first substituting in \eqref{eq: phi-PDF}
\begin{align}
     \dfrac{-\nu}{\sqrt{2\pi}} \;\int_{-\inf}^{d_{-}} ye^{-y^2} = \dfrac{\nu}{\sqrt{2\pi}} \;e^{\frac{-d_-^2}{2}}
\end{align}
Thus, our final simplified form is
\begin{align}
     V_0 = \tilde{\E}[C_\tau] + \dfrac{\nu}{\sqrt{2\pi}} \;e^{\frac{-d_-^2}{2}} +  (K- S_0 - \mu) \Phi(d_-) 
\end{align}

(FIXME: continue with this computation, write down an explicit formula for $V_0$, and also the price for the variant.) (FOLLOWUP still have to write up variant)

\subsection*{Acknowledgement} The authors would like to thank Prof. Hrusa for his patient guidance on this project.  


\appendix
\section{Various replicating strategies}

\subsection{Replicating European options}

\subsection{Replicating Asian options}

%\printbibliography













\end{document}









