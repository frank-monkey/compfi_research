
\section{Blachelier Model - the continuous case}
\subsection{The case when $r = 0$}
\subsubsection{European Call}
We consider the Blachelier model where the stock price evolves according to 
\begin{align}
	 S_t = S_0 + \kappa W_t,
\end{align}
where $\{W_t\}_{t \ge 0}$ is a Brownian motion under the risk neutral measure $\tilde{\mathbb{P}}$. We note that under $\tilde{\mathbb{P}}$, $ W_T \sim N(0, T)$, therefore
\begin{align}\label{eq: S_T dist}
S_T \sim N(S_0, \kappa^2T) \; \text{under the risk neutral measure} \; \tilde{\mathbb{P}}.
\end{align}
Consider a European call where the payoff at time $T$ is given by 
\begin{align}
	 V_T = (S_T - K)^+.
\end{align}
According to the risk neutral pricing formula, the time-0 price of this security is given by 
\begin{align}
	 V_0 = \tilde{\E}[(S_T - K)^+].
\end{align}
Recall that if we have a random variable $X$ with probability density function $f_X$ under a probability measure $\mathbb{P}$, then the ``law of the unconscious statistician'' tells us that 
\begin{align}
	 \E[g(X)] = \int_{-\infty}^\infty g(x) f_X(x) \; dx.
\end{align}
In our setting, we have 
\begin{align}
	 g(S_T) = (S_T - K)^+,
\end{align}
and the distribution of $S_T$ under $\tilde{\bP}$ as a random variable is given in \eqref{eq: S_T dist}. Therefore 
\begin{align}\label{eq: call int}
	 V_0 = \tilde{\E}[(S_T - K)^+] =  \tilde{\E}[g(S_T)] = \int_{-\infty}^\infty (x-K)^+ \psi(x) \; dx,
\end{align}
where 
\begin{align}
	 \psi(x) = \frac{1}{\nu}\varphi\left(\frac{x-\mu}{\nu}\right), \; \varphi(y) = \frac{1}{\sqrt{2\pi}}e^{-y^2/2}, 
\end{align}
and
\begin{align}
	 \mu = S_0, \; \nu = \kappa \sqrt{T}.
\end{align}
To compute \eqref{eq: call int}, we first note that since $(x-K)^+ = 0$ for $x \le K$, the domain of integration is the set $\{x \mid x \ge K\}$. Now we use the change of variables 
\begin{align}
	 y = -\frac{x-\mu}{\nu} \Longleftrightarrow x = \mu - \nu y
\end{align}
and we note that since $\nu > 0$, 
\begin{align}
	 x \ge K \Longleftrightarrow \frac{x-\mu}{\nu} \ge \frac{K - \mu}{\nu} \Longleftrightarrow y \le \frac{\mu - K}{\nu} =: d_-.
\end{align}
Then by performing a change of variables, \eqref{eq: call int} becomes 
\begin{align}
	 V_0 = \int_{-\infty}^{d_-} (\nu y + K - \mu) \varphi(-y) \; dy = \int_{-\infty}^{d_-} (\nu y + K - \mu) \varphi(y) \; dy = \underbrace{\int_{-\infty}^{d_-} \nu y \varphi(y) \; dy}_{:= I} + \underbrace{\int_{-\infty}^{d_-}  (K-\mu)\varphi(y) \; dy}_{:= II}.
\end{align}
We define the cumulative distribution function of a standard normal random variable $X$ under $\mathbb{P}$ via 
\begin{align}
	 \Phi(x) = \mathbb{P}[X \le x] = \E[ \mathbbm{1}_{X \le x}] = \int_{-\infty}^x \varphi(y) \; dy.
\end{align}
With this notation in hand, we can write 
\begin{align}
	 II = (K-\mu) \int_{-\infty}^{d_-} \varphi(y) \; dy = (K-\mu) \Phi(d_-),
\end{align}
and 
\begin{align}
	 I = \nu \int_{-\infty}^{d_-} y \varphi(y) \; dy =  \frac{\nu}{\sqrt{2\pi}} \lim_{t \to -\infty} (e^{-t^2/2} - e^{-d_-^2/2}) = -\frac{\nu}{\sqrt{2\pi}} e^{-d_-^2/2}.
\end{align}
Therefore 
\begin{align}
	 V_0 =  -\frac{\nu}{\sqrt{2\pi}} e^{-d_-^2/2} + (K-\mu) \Phi(d_-).
\end{align}
To compute the price of a put, one can use put-call parity. 
\subsubsection{Arithmetic Asian call}
Consider an arithmetic Asian call where the payoff at time $T$ is given by 
\begin{align}
	 V_T = (A_T - K)^T, \; A_T = \frac{1}{T}\int_0^T S_t \; dt = S_0 + \frac{\kappa}{T} \int_0^T W_t \; dt. 
\end{align}
Using Ito's formula, one can show that under the risk neutral measure $\tilde{\mathbb{P}}$, 
\begin{align}
	 \int_0^T W_t \; dt \sim N(0, T^3/3).
\end{align}
Therefore we have 
\begin{align}
	 A_T \sim N(S_0, \kappa^2 T/3) \; \text{under the risk neutral measure} \; \tilde{\mathbb{P}}.
\end{align}
Comparing this to \eqref{eq: S_T dist}, we see that $A_T$ has a similar distribution, the only difference is that the variance of $A_T$ is smaller by a factor of $1/3$, so the standard deviation of $A_T$ is smaller by a factor of $1/\sqrt{3}$. By performing the exact same set of calculation as before, the time-0 price of an Asian option is 
\begin{align}\label{eq: asian op}
	 V_0 = -\frac{\nu}{\sqrt{3}\sqrt{2\pi}} e^{-3d_-^2/2} + (K-\mu) \Phi(\sqrt{3}d_-),
\end{align}
where 
\begin{align}
	 \mu = S_0, \; \nu = \kappa \sqrt{T}, \; d_- = \frac{\mu - K}{\nu}.
\end{align}
We note that since $\sqrt{3} > 1$, we see from \eqref{eq: asian op} that the price of an Asian option is higher than the price of a European call. This should be expected as one is paying a premium for a less volatile product. 

\subsection{The case when $r \neq 0$}
Now we turn our attention to the case when $r \neq 0$. Here, the stock prices evolve according to 
\begin{align}
	 S_t = e^{rt} \left( S_0 + \kappa^{-rt} W_t + \kappa r \int_0^t e^{-rs} W_s \; ds \right).
\end{align}
In order to perform the same set of calculations as above, we need to understand the distribution of 
\begin{align}
	 S_T = e^{rT} \left( S_0 + \kappa^{-rT} W_T + \kappa r \int_0^T e^{-rt} W_t \; dt \right)
\end{align}
under $\tilde{\mathbb{P}}$. Using Ito's formula, one can show that if $H$ is continuously differentiable, then under the risk neutral measure $\tilde{\mathbb{P}}$, 
\begin{align}
	 \int_0^T H'(t) W_t \; dt
\end{align}
is normally distributed with mean 0 and variance 
\begin{align}
	 \int_0^T (H(T) - H(t))^2 \; dt.
\end{align}
In our setting, we set 
\begin{align}
	 H(t) = \frac{-e^{-rt}}{r},
\end{align} 
and we calculate 
\begin{multline}
	 \int_0^T (H(T) - H(t))^2 \; dt = \frac{1}{r^2} \int_0^T (e^{-rT} - e^{-rt})^2 \; dt = \frac{1}{r^2} \int_0^T (e^{-2rT} - 2e^{-rT}e^{-rt} + e^{-2rt}) \; dt \\= \frac{1}{r^2} \left( T e^{-2rT} - \frac{2 e^{-rT}}{r} (1-e^{-rT}) + \frac{1}{2r} (1-e^{-2rT}) \right).
\end{multline}
Then 
\begin{align}
	 \tilde{\E}[S_T] = e^{rT} S_0 =: \mu
\end{align}
and 
\begin{multline}
	 \tilde{\mathbb{V}}[S_T] = e^{2rT} \left(\kappa^{-2rT}T + \kappa^2 \left( T e^{-2rT} - \frac{2 e^{-rT}}{r} (1-e^{-rT}) + \frac{1}{2r} (1-e^{-2rT}) \right)\right) \\
	 = (e/\kappa)^{2rT} T + \kappa^2 \left(T - \frac{2}{r} e^{rT} + \frac{2}{r} + \frac{1}{2r} e^{2rT} - \frac{1}{2r} \right) := \sigma^2
\end{multline}
(Is there a better way to write out the variance? I'm not sure.) Now that we know $S_T \sim N(\mu,\sigma^2)$, we can find the time-0 price of a European call using the risk neutral pricing formula 
\begin{align}
	 V_0 = \tilde{\E}[e^{-rT} (S_T - K)^+],
\end{align}
and we perform the same set of calculations as in the case when $r = 0$. A good exercise is for us to do the computations and see if we can get a relatively clean formula.

\section{Blachelier Model - the binomial case}
Now we turn our attention to the binomial case. For $n = 0, \ldots , N-1$, the stock prices evolve according to 
\begin{align}
	 S_{n+1} - S_n = \mu S_n + \beta X_{n+1},
\end{align}
where $X_k$ is a random variable satisfying 
\begin{align}
	 X_k = \begin{cases}
		 1 & \text{if} \; \omega_k = H \\
		 -1 & \text{if} \; \omega_k = T.
	 \end{cases}
\end{align}
Here the stock prices satisfy a set of finite difference equations, so it's likely that there is a closed form expression for $S_n$. To find this closed form, we use the ansatz 
\begin{align}\label{eq: ansatz}
	 S_n = (1+\mu)^n f_n.
\end{align}
Then we note that $f_n$ satisfies 
\begin{align}
	 f_{n+1} - f_n = \frac{\beta}{(1+\mu)^{n+1}} X_{n+1}, \quad n = 0, \ldots , N-1.
\end{align}
We note that if we add up the first $k$ equations, then on the LHS we have a telescoping sum of the form 
\begin{align}
	 \sum_{n=0}^{k-1} (f_{n+1} - f_n) = f_k - f_0.
\end{align}
Therefore, we find that for any $0 \le k \le N$, 
\begin{align}
	 f_k = f_0 + \sum_{j=0}^{k-1} \frac{\beta}{(1+\mu)^{j+1}} X_{j+1}, 
\end{align}
Once we have a closed form expression for $f_n$, we can recover a closed form expression for $S_n$ via \eqref{eq: ansatz}. We should have 
\begin{align}
	 S_n =(1+\mu)^n f_n = (1+\mu)^n\left( S_0  + \sum_{j=0}^{n-1} \beta (1+\mu)^{-j-1} X_{j+1} \right).
\end{align}
Here we see explicitly that $S_n$ depends on all first $n$ coin tosses. We note that if $m \neq n$, then $X_m, X_n$ are independent random variables, as the value of $X_k$ only depends on the $k$-th coin flip. Furthermore, 
\begin{multline}
	 S_m S_n = (1+\mu)^{m+n} \bigg( S_0^2 +  S_0 \left( \sum_{j=0}^{m-1} \beta (1+\mu)^{-j-1} X_{j+1}  + \sum_{j=0}^{n-1} \beta (1+\mu)^{-j-1} X_{j+1}  \right) \\ +   \sum_{j=0}^{m-1}  \sum_{k=0}^{n-1} \beta^2 (1+\mu)^{-j-k-2} X_{j+1} X_{k+1}\bigg).
\end{multline}
So by linearity of expectation we should have 
\begin{align}
	 \tilde{\E}[S_m S_n] = \tilde{\E}[S_m] \tilde{\E}[S_n].
\end{align}
So $S_m, S_n$ are independent under $\tilde{\mathbb{P}}$ for $m \neq n$, though they are not identically distributed under $\tilde{\mathbb{P}}$ as each $S_n$ is a constant term depending on $n$ plus a sum of random variables that are Bernoulli distributed with different mean and variances under $\tilde{\mathbb{P}}$. The means and variances are different as the up and down factors are path dependent. We can see this explicitly by nothing that 
\begin{align}
	 \frac{S_{n+1}}{S_n} = \frac{(1+\mu)S_{n} + \beta X_{n+1}}{S_n} = (1+\mu) + \beta \frac{X_{n+1}}{S_n},
\end{align}
so the up and down factors depend on all $n+1$ coin tosses, as $X_{n+1}$ depends on the $n+1$-th coin toss and $S_n$ depends on the first $n$. 

In term, this implies that the risk neutral probabilities 
\begin{align}
	 \tilde{p}_n = \frac{1 + r - d_n}{u_n-d_n}, \quad \tilde{q}_n = 1 - \tilde{p}_n = \frac{u_n - (1+r)}{u_n-d_n}
\end{align}
at each node are path dependent. 

\subsection{Geometric Asian options}
Consider a geometric Asian call where the payoff at time $T$ is 
\begin{align}
	 V_T = (G_T - K)^+, 
\end{align}
where 
\begin{align}
	 G_T = \left( \prod_{i=1}^T S_i \right)^{1/T}.
\end{align}
If we assume that the stock prices are never negative, then we can also write $G_T$ as 
\begin{align}
	 G_T = \exp \left( \frac{1}{T} \sum_{i=1}^T \ln S_i \right).
\end{align}
In the standard Black-Scholes model, in the continuous case $S_t$ is log-normally distributed under the risk neutral measure. As a consequence, $G_T$ will also be log-normally distributed under the risk neutral measure, so there are ways to write down explicit formulas for prices of a geometric Asian call. In our setting, however, it's not immediate clear how $G_T$ is distributed under $\tilde{\mathbb{P}}$, so we need to look for other means of computing $V_0 = \tilde{\E}[(G_T - K)^+]$. 

First we'll consider a simpler payoff, say $V_T = G_T$. One possibility of computing $V_0$ is to look for an underlying Markov process in the model. In the standard up and down model, we can write $S_{n+1} = S_{n} \frac{S_{n+1}}{S_n}$, where $S_n$ depends only on the first $n$ coin tosses and $S_{n+1}/S_n$ only depends on the $n+1$-th coin toss. This implies that $S_n$ is Markov, but in our setting $S_{n+1}/S_{n}$ depends on all $n+1$ tosses. 

First note that the up and down factors in our setting can be written as a function of $S_n$, since 
\begin{align}
	 \frac{S_{n+1}}{S_n} = (1+\mu) + \beta \frac{X_{n+1}}{S_n}.
\end{align}
So the up and down factors at time $n$ can be written as $u_n(S_n), d_n(S_n)$, where 
\begin{align}
	 u_n(s) = (1+ \mu) + \frac{\beta}{s}, \quad d_n(s) = (1+\mu) - \frac{\beta}{s}.
\end{align}
Then $\tilde{p}_n, \tilde{q}_n$ can also be written as $\tilde{p}_n(S_n), \tilde{q}_n(S_n)$, where 
\begin{align}
	 \tilde{p}_n(s) = \frac{1 + r - d_n(s)}{u_n(s) - d_n(s)} = \frac{ \beta + s( r-\mu) }{2\beta}, \quad \tilde{q}_n (s) = \frac{ \beta -  s( r-\mu) }{2\beta}.
\end{align}
Furthermore, we can write
\begin{align}
	 G_{n+1}^{n+1} = S_{n+1} G_n^{n} = (\beta X_{n+1} + (1+\mu)S_n) G_n^n = \beta X_{n+1} G_n^n + (1+\mu)S_n G_n^n,
\end{align}
where $X_{n+1}$ only depends on the $(n+1)$-th coin toss and $S_n, G_n$ only depend on the first $n$ coin tosses. 

Then we have 
\begin{align}
	 \tilde{\E}_n[G_{n+1}] = \tilde{\E}_n[(\beta X_{n+1} G_n^n + (1+\mu)S_n G_n^n)^{1/(n+1)} ] = g(S_n, G_n),
\end{align}
where 
\begin{multline}
	 g(s,g) = \tilde{\E}_n [ (\beta X_{n+1} g^n + (1+\mu) sg^n)^{1/(n+1)} ] = \tilde{p}_n(s) ((1+\mu)sg^n + \beta g^n)^{1/(n+1)} \\ + \tilde{q}_n(s) ((1+\mu)sg^n - \beta g^n)^{1/(n+1)}.
\end{multline}
So If $V_N = G_N$ and by the risk neutral pricing formula we have
\begin{align}
	 V_n = \frac{1}{1+r}\tilde{\E}_n[ V_{n+1}], \quad 
\end{align}
for $n = 0, \ldots , N-1$, then $V_n = v_n(S_n ,G_n)$ where 
\begin{align}
	 &v_N (s,g) = g \\
	 &v_n (s,g) =\frac{1}{1+r} \left[ \frac{ \beta + s( r-\mu) }{2\beta} ((1+\mu)sg^n + \beta g^n)^{1/(n+1)} + \frac{ \beta -  s( r-\mu) }{2\beta}((1+\mu)sg^n - \beta g^n)^{1/(n+1)} \right], 
\end{align}
for $n= 0, \ldots , N-1$.

% One possibility of approaching this problem is by considering the random variable 
% \begin{align}
% 	 Y_n = \prod_{i=1}^n S_i,
% \end{align}
% and we can try to show that $(S_n, Y_n)$ is Markov under $\tilde{\mathbb{P}}$. This means that for every function $f$ and $n = 1, \ldots, N$, we want to show that there exists a deterministic function $g$ for which 
% \begin{align}
% 	 \tilde{\E}_n[f(S_{n+1}, Y_{n+1})] = g(S_n, Y_n).
% \end{align}
% In our setting, we note that 
% \begin{align}
% 	 S_{n+1} = S_n \frac{S_{n+1}}{S_n}, \; Y_{n+1} = S_{n+1} Y_n = S_n Y_n \frac{S_{n+1}}{S_n},
% \end{align}
% where $S_n, Y_n$ only depends on the first $n$ coin tosses and \sout{$S_{n+1}/S_n$ only depends on the $n+1$-th toss} (this is not true in our setting because the up/down factors depend on $S_n$, shoot. This means that everything after this line is wrong). Then by using the independence lemma (a generalized way of ``taking out what is known''), for any deterministic function $f_{n+1}(s,y)$, we have 
% \begin{align}
% 	 \tilde{\E}_n[f_{n+1}(S_{n+1}, Y_{n+1})] = \tilde{\E}_n\left[f_{n+1}\left(S_n \frac{S_{n+1}}{S_n},S_n Y_n \frac{S_{n+1}}{S_n} \right)\right] = h_n(S_n, Y_n),
% \end{align}
% and by using the fact that $S_{n+1}/S_n$ depends only on the $n+1$-th coin toss, 
% \begin{align}\label{eq: hn}
% 	 h_n(s,y) = \tilde{\E}_n\left[ f_{n+1} \left(s \frac{S_{n+1}}{S_n},sy \frac{S_{n+1}}{S_n} \right) \right] =  \tilde{\E}\left[ f_{n+1} \left(s \frac{S_{n+1}}{S_n},sy \frac{S_{n+1}}{S_n} \right) \right] = \tilde{p}_n f_{n+1}(s u_n , sy u_n) + \tilde{q}_n f_{n+1}(s d_n, sy d_n).
% \end{align}
% Normally we can conclude that $\{(S_n, Y_n)\}_{n=1}^N$ is Markov, but note that  $\tilde{p}_n, \tilde{q}_n, u_n, d_n$ are not deterministic, they are random. 

% As a consequence, the price $V_n$ of the price of a geometric Asian call is a deterministic function $v_n$ of $S_n$ and $Y_n$, i.e. 
% \begin{align}\label{eq: Vn}
% 	 V_n(\omega) = v_n(S_n (\omega), Y_n(\omega)), \quad n=0,\ldots, N.
% \end{align}
% Here, we have a terminal boundary condition 
% \begin{align}\label{eq: bdd}
% v_N(s,y) = (y - K)^+.
% \end{align}
% Combining \eqref{eq: hn}, \eqref{eq: Vn}, and the risk neutral pricing formula, for $n= 0 , \ldots , N-1$ we should have 
% \begin{align}\label{eq: back}
% 	 v_n(s,y) = \frac{1}{1+r}[ \tilde{p}_n v_{n+1}(s u_n , sy u_n) + \tilde{q}_n v_{n+1}(s d_n, sy d_n)]
% \end{align}
% So to calculate $V_0$, we do backwards induction using the terminal boundary condition \eqref{eq: bdd} and equations \eqref{eq: back}. 

% If the final payoff is a bit nicer, for example if $V_T = Y_T$, then the terminal condition becomes 
% \begin{align}
% 	 V_N(s,y) = y.
% \end{align}
% So we might be able to solve the finite difference equations explicitly to get a closed form expression for $V_0$. 